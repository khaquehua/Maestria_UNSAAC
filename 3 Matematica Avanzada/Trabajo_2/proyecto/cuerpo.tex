
%\tableofcontents
\newpage

%%%%%%%%%%%%%%%%%%%%%%
%%%%%%%%% Capítulo I
%%%%%%%%%%%%%%%%%%%%%%%

\section{Problema}

Demuestre que $S$ es un subespacio de $\mathbb{R}^3$, donde

\[
S = \left\{ \mathbf{x} \in \mathbb{R}^3 : \mathbf{x} = \alpha
\begin{bmatrix}
1 \\ 
2 \\ 
3
\end{bmatrix}
+ \beta
\begin{bmatrix}
2 \\ 
1 \\ 
0
\end{bmatrix}, \, \alpha, \beta \in \mathbb{R} \right\}.
\]

\subsection{Solución}
Para demostrar que $S$ es un subespacio vectorial de $\mathbb{R}^3$, comprobemos que cumpla la siguiente definición

\begin{defi}\textbf{Espacio vectorial en $\mathbb{R}^n$}
    \\
    Un conjunto $S$ no vacío de $\mathbb{R}^n$, es llamado un espacio vectorial en $\mathbb{R}^n$ si verifica las siguientes condiciones
    \begin{itemize}
        \item[i)] Para $\mathbf{x},\mathbf{y} \in S$, entonces $\mathbf{x}+\mathbf{y} \in S$ 
        \item[ii)] Para $\alpha \in \mathbb{R}$, $\mathbf{x} \in S$, entonces $\alpha \mathbf{x} \in S$ 
    \end{itemize}
\end{defi}

\subsubsection{Primera condición}
Sean $\mathbf{x}$,$\mathbf{y} \in S$, expresados de la siguiente manera:

\[
\mathbf{x} = {\alpha}_{1} 
\begin{bmatrix}
1 \\ 
2 \\ 
3
\end{bmatrix}
+
{\beta}_{1}
\begin{bmatrix}
2 \\ 
1 \\ 
0
\end{bmatrix},
\qquad
\mathbf{y} = {\alpha}_{2} 
\begin{bmatrix}
1 \\ 
2 \\ 
3
\end{bmatrix}
+
{\beta}_{2}
\begin{bmatrix}
2 \\ 
1 \\ 
0
\end{bmatrix},
\]

Donde ${\alpha}_{1},{\alpha}_{2},{\beta}_{1},{\beta}_{2} \in \mathbb{R}$, realizemos la siguiente operación $\mathbf{x}+\mathbf{y}$ el cual debe pertenecer a $S$

\[
\mathbf{x} + \mathbf{y} = \left( {\alpha}_{1} 
\begin{bmatrix}
1 \\ 
2 \\ 
3
\end{bmatrix}
+
{\beta}_{1}
\begin{bmatrix}
2 \\ 
1 \\ 
0
\end{bmatrix} \right)
+
\left( {\alpha}_{2} 
\begin{bmatrix}
1 \\ 
2 \\ 
3
\end{bmatrix}
+
{\beta}_{2}
\begin{bmatrix}
2 \\ 
1 \\ 
0
\end{bmatrix} \right)
\]

En esta parte se puede agrupar los vectores $\begin{bmatrix}
1 \\ 
2 \\ 
3
\end{bmatrix}$ y
$
\begin{bmatrix}
2 \\ 
1 \\ 
0
\end{bmatrix}$ como términos comúnes

\[
\mathbf{x} + \mathbf{y} = ({\alpha}_{1}+{\alpha}_{2}) 
\begin{bmatrix}
1 \\ 
2 \\ 
3
\end{bmatrix}
+
({\beta}_{1}+{\beta}_{2})
\begin{bmatrix}
2 \\ 
1 \\ 
0
\end{bmatrix}
\]

Se sabe que si ${\alpha}_{1},{\alpha}_{2},{\beta}_{1},{\beta}_{2} \in \mathbb{R}$ entonces ${\alpha}_{1}+{\alpha}_{2} \in \mathbb{R}$, de igual forma ${\beta}_{1}+{\beta}_{2} \in \mathbb{R}$, entonces se cumple que $\mathbf{x} +\mathbf{y} \in S$

\subsubsection{Segunda condición}
Sea $\mathbf{x} \in S$, esto es

\[
\mathbf{x} = \alpha 
\begin{bmatrix}
1 \\ 
2 \\ 
3
\end{bmatrix}
+
\beta
\begin{bmatrix}
2 \\ 
1 \\ 
0
\end{bmatrix}, \alpha , \beta \in \mathbb{R}
\]

Multipliquemos por un escalar $\gamma \in \mathbb{R}$:

\[
\gamma \mathbf{x} = \gamma \left( \alpha 
\begin{bmatrix}
1 \\ 
2 \\ 
3
\end{bmatrix}
+
\beta
\begin{bmatrix}
2 \\ 
1 \\ 
0
\end{bmatrix} \right)
\]

\[
\gamma \mathbf{x} = \gamma \alpha 
\begin{bmatrix}
1 \\ 
2 \\ 
3
\end{bmatrix}
+
\gamma \beta
\begin{bmatrix}
2 \\ 
1 \\ 
0
\end{bmatrix}
\]

Se sabe que si ${\alpha},{\beta},{\gamma} \in \mathbb{R}$ entonces ${\alpha}{\gamma} \in \mathbb{R}$, de igual forma ${\beta}{\gamma} \in \mathbb{R}$, entonces se cumple que $\gamma \mathbf{x} \in S$ \\

\textbf{CONCLUSIÓN: }$S$ es un subespacio vectorial de $\mathbb{R}^3$

\section{Problema}

Considere el siguiente conjunto de vectores en $\mathbb{R}^3$:

\[
X = \left\{ 
\begin{bmatrix}
1 \\ 
0 \\ 
-1
\end{bmatrix}
,
\begin{bmatrix}
-2 \\ 
1 \\ 
1
\end{bmatrix} \right\}.
\]

Demuestre que uno de los dos vectores

\[
\mathbf{u} = 
\begin{bmatrix}
-1 \\ 
2 \\ 
1
\end{bmatrix}
\quad
\text{y}
\quad
\mathbf{v} = 
\begin{bmatrix}
-1 \\ 
1 \\ 
1
\end{bmatrix}
\]

pertenece a $gen(X)$, mientras que el otro no.\\
Encuentre un número real de $\lambda$ tal que

\[
\begin{bmatrix}
1 \\ 
1 \\ 
\lambda
\end{bmatrix}
\in gen(X)
\]

\subsection{Solución}
Para demostrar que uno de los $\mathbf{u},\mathbf{v} \in gen(X)$ se debe cumplir la siguiente definición

\begin{defi}\textbf{Espacio generado por los vectores}
    \\
    El conjunto de todas las posibles combinaciones lineales de ${v}_{1} , {v}_{2} , \cdots , {v}_{r}$ es llamado espacio generado $gen(X)$ por ${v}_{1} , \cdots , {v}_{n}$ si
    $$
    gen({v}_{1} , \cdots , {v}_{n} ) = \{ {\alpha}_{1} {v}_{1} + \cdots + {\alpha}_{r} {v}_{r} / {\alpha}_{i} \in \mathbb{R} , i = 1 , \cdots , r \}
    $$
\end{defi}
\subsubsection{Para el vector $\mathbf{u}$}
Se debe cumplir la siguiente combinación lineal
\[
\mathbf{u} = {\alpha}_{1}
\begin{bmatrix}
1 \\ 
0 \\ 
-1
\end{bmatrix}
+ {\alpha}_{2}
\begin{bmatrix}
-2 \\ 
1 \\ 
1
\end{bmatrix}
\]

Reemplanzado por el vector $\mathbf{u}$ se tiene la siguiente igualdad
\[
\begin{bmatrix}
-1 \\ 
2 \\ 
1
\end{bmatrix}
= {\alpha}_{1}
\begin{bmatrix}
1 \\ 
0 \\ 
-1
\end{bmatrix}
+ {\alpha}_{2}
\begin{bmatrix}
-2 \\ 
1 \\ 
1
\end{bmatrix}
\]

Se tienen las siguientes combinaciones lineales
\begin{eqnarray}
    {\alpha}_{1} - 2 {\alpha}_{2} &=& -1 \\
    {\alpha}_{2} &=& 2 \\
    -{\alpha}_{1} + {\alpha}_{2} &=& 1
\end{eqnarray}
Del cual de (2) se tiene que $\alpha_2 = 2$, reemplazemos estos valores en (1) y (3):
\begin{itemize}
     \item Reemplanzado en (1):
     \begin{eqnarray}
         {\alpha}_{1} - 2 {\alpha}_{2} &=& -1 \nonumber \\
         {\alpha}_{1} - 2 (2) &=& -1 \nonumber \\
         {\alpha}_{1} &=& 3 \nonumber
     \end{eqnarray}
     Se observa que ${\alpha}_{1} = 3$, y cumple la igualdad en $\mathbf{u}$, ahora reemplazemos ${\alpha}_{1} = 3$ y ${\alpha}_{2} = 2$ en (3) 
     \item Reemplanzado en (3):
     \begin{eqnarray}
         -{\alpha}_{1} + {\alpha}_{2} &=& 1 \nonumber \\
         -(3) + (2) &=& -1 \neq 1 \nonumber
     \end{eqnarray}
     Se observa que no cumple la igualdad.
 \end{itemize} 
Por lo tanto $\mathbf{u} \notin gen(X)$

\subsubsection{Para el vector $\mathbf{v}$}
Se debe cumplir la siguiente combinación lineal
\[
\mathbf{v} = {\alpha}_{1}
\begin{bmatrix}
1 \\ 
0 \\ 
-1
\end{bmatrix}
+ {\alpha}_{2}
\begin{bmatrix}
-2 \\ 
1 \\ 
1
\end{bmatrix}
\]

Reemplanzado por el vector $\mathbf{v}$ se tiene la siguiente igualdad
\[
\begin{bmatrix}
-1 \\ 
1 \\ 
1
\end{bmatrix}
= {\alpha}_{1}
\begin{bmatrix}
1 \\ 
0 \\ 
-1
\end{bmatrix}
+ {\alpha}_{2}
\begin{bmatrix}
-2 \\ 
1 \\ 
1
\end{bmatrix}
\]
\setcounter{equation}{0}
Se tienen las siguientes combinaciones lineales
\begin{eqnarray}
    {\alpha}_{1} - 2 {\alpha}_{2} &=& -1 \\
    {\alpha}_{2} &=& 1 \\
    -{\alpha}_{1} + {\alpha}_{2} &=& 1
\end{eqnarray}
Del cual de (2) se tiene que $\alpha_2 = 1$, reemplazemos estos valores en (1) y (3):
\begin{itemize}
     \item Reemplanzado en (1):
     \begin{eqnarray}
         {\alpha}_{1} - 2 {\alpha}_{2} &=& -1 \nonumber \\
         {\alpha}_{1} - 2 (1) &=& -1 \nonumber \\
         {\alpha}_{1} &=& 1 \nonumber
     \end{eqnarray}
     Se observa que ${\alpha}_{1} = 1$, y cumple la igualdad en $\mathbf{v}$, ahora reemplazemos ${\alpha}_{1} = 1$ y ${\alpha}_{2} = 1$ en (3) 
     \item Reemplanzado en (3):
     \begin{eqnarray}
         -{\alpha}_{1} + {\alpha}_{2} &=& 1 \nonumber \\
         -(1) + (1) &=& 0 \neq 1 \nonumber
     \end{eqnarray}
     Se observa que no cumple la igualdad.\\
     Por lo tanto $\mathbf{v} \notin gen(X)$
 \end{itemize} 

\textbf{CONCLUSIÓN:} Ninguno de los vectores $\mathbf{u},\mathbf{v} \notin gen(X)$

\subsection{Encontrar un número real $\lambda$}
De igual forma igualamos la combinación lineal e igualamos al vector para hallar el valor de $\lambda$

\setcounter{equation}{0}
Se tienen las siguientes combinaciones lineales
\begin{eqnarray}
    {\alpha}_{1} - 2 {\alpha}_{2} &=& 1 \\
    {\alpha}_{2} &=& 1 \\
    -{\alpha}_{1} + {\alpha}_{2} &=& \lambda
\end{eqnarray}
Del cual de (2) se tiene que $\alpha_2 = 1$, reemplazemos estos valores en (1) y (3):
\begin{itemize}
     \item Reemplanzado en (1):
     \begin{eqnarray}
         {\alpha}_{1} - 2 {\alpha}_{2} &=& 1 \nonumber \\
         {\alpha}_{1} - 2 (1) &=& 1 \nonumber \\
         {\alpha}_{1} &=& 3 \nonumber
     \end{eqnarray}
     Se observa que ${\alpha}_{1} = 3$ y ${\alpha}_{2} = 1$, reemplazemos estos valores en (3) 
     \item Reemplanzado en (3):
     \begin{eqnarray}
         -{\alpha}_{1} + {\alpha}_{2} &=& \lambda \nonumber \\
         -(3) + (1) &=& \lambda \nonumber \\
         \lambda &=& -2 \nonumber
     \end{eqnarray}
     El valor de $\lambda$ para que el vector $\begin{bmatrix}
1 \\ 
1 \\ 
\lambda
\end{bmatrix} \in gen(X)$ es -2 
 \end{itemize} 

\textbf{CONCLUSIÓN:} El número real de $\lambda$ es -2

\section{Problema}

Verifique si cada uno de los siguientes conjuntos de vectores son linealmente independientes:

\begin{enumerate}[label=\alph*)] % Cambia el formato a a), b), c), etc.
    \item 

    \[
    \left\{ 
    \begin{bmatrix}
    1 \\ 
    2 \\ 
    3
    \end{bmatrix}
    ,
    \begin{bmatrix}
    1 \\ 
    0 \\ 
    -1
    \end{bmatrix}
    ,
    \begin{bmatrix}
    -2 \\ 
    1 \\ 
    1
    \end{bmatrix} \right\}.
    \]

    \item 
    
    \[
    \left\{ 
    \begin{bmatrix}
    1 \\ 
    2 \\ 
    3
    \end{bmatrix}
    ,
    \begin{bmatrix}
    4 \\ 
    5 \\ 
    6
    \end{bmatrix}
    ,
    \begin{bmatrix}
    7 \\ 
    8 \\ 
    9
    \end{bmatrix} \right\}.
    \]

    \item

    \[
    \left\{ 
    \begin{bmatrix}
    1 \\ 
    2 \\ 
    3 \\
    4
    \end{bmatrix}
    ,
    \begin{bmatrix}
    1 \\ 
    1 \\ 
    1 \\
    1
    \end{bmatrix}
    ,
    \begin{bmatrix}
    1 \\ 
    3 \\ 
    1 \\
    2
    \end{bmatrix}
    ,
    \begin{bmatrix}
    2 \\ 
    1 \\ 
    1 \\
    2
    \end{bmatrix} \right\}.
    \]

    \item

    \[
    \left\{ 
    \begin{bmatrix}
    1 \\ 
    2 \\ 
    -1 \\
    0
    \end{bmatrix}
    ,
    \begin{bmatrix}
    0 \\ 
    1 \\ 
    2 \\
    1
    \end{bmatrix}
    ,
    \begin{bmatrix}
    2 \\ 
    3 \\ 
    -4 \\
    -1
    \end{bmatrix} \right\}.
    \]

\end{enumerate}

\subsection{Solución}

Para verificar que uno de los conjuntos de vectores son linealmente independientes de la definición de \textbf{Independencia lineal} tomamos los siguientes enunciados:
\begin{itemize}
    \item El conjunto $A = \{ {v}_{1}, \cdots , {v}_{r} \}$ de vectores ${v}_{i} \in \mathbb{R}^n$, es llamado conjunto linealmente independiente si los vectores ${v}_{1}, \cdots , {v}_{r}$ son linealmente independientes.
    \item Sea $A = \{ {v}_{1}, \cdots , {v}_{r} \} \subset \mathbb{R}^n$ y sea $A = [ {v}_{1} \quad {v}_{2} \quad \cdots \quad {v}_{r} ]$ una matriz de orden $nxr$. $A$ es linealmente independiente $\iff N(A)=\{ 0 \}$
\end{itemize}

\subsubsection{Para a)}
Tenemos los siguientes vectores
\begin{itemize}
    \item ${v}_{1} = (1,2,3)$
    \item ${v}_{2} = (1,0,-1)$
    \item ${v}_{3} = (-2,1,1)$ 
\end{itemize}
y el conjunto $A=\{ {v}_{1}, {v}_{2}, {v}_{3} \}$ expresando el conjunto en una matriz quedaría de la siguiente forma

\[
A = [ {v}_{1} \quad {v}_{2} \quad {v}_{3} ] =
\begin{bmatrix}
1 & 1 & -2 \\ 
2 & 0 & 1 \\
3 & -1 & 1
\end{bmatrix}
\Longrightarrow N(A) = \{ \mathbf{0} \} ?
\]
$$
N(A) = \{ \mathbf{x} \in \mathbb{R}^3 / A \mathbf{x} = \mathbf{0} \}
$$
La solución del sistema lineal estaría representado de la siguiente manera:
\[
\begin{bmatrix}
1 & 1 & -2 \\ 
2 & 0 & 1 \\
3 & -1 & 1
\end{bmatrix}
\begin{bmatrix}
{x}_{1} \\ 
{x}_{2} \\
{x}_{3}
\end{bmatrix} =
\begin{bmatrix}
0 \\ 
0 \\
0
\end{bmatrix}
\]
El cual nos da el siguiente sistema de combinaciones lineales
\setcounter{equation}{0}
\begin{eqnarray}
    {x}_{1} + {x}_{2} - 2{x}_{3} &=& 0 \\
    2{x}_{1} + {x}_{3} &=& 0 \\
    3{x}_{1} - {x}_{2} + {x}_{3} &=& 0
\end{eqnarray}
De lo cual en la ecuación (2) se tiene que ${x}_{3} = -2{x}_{1}$, si reemplazamos este valor en la ecuación (1) se tiene
\begin{eqnarray}
    {x}_{1} + {x}_{2} - 2{x}_{3} &=& 0 \nonumber \\
    {x}_{1} + {x}_{2} - 2(-2{x}_{1}) &=& 0 \nonumber \\
    {x}_{1} + {x}_{2} + 4{x}_{1} &=& 0 \nonumber \\
    {x}_{2} &=& -5{x}_{1} \nonumber
\end{eqnarray}
Reemplanzado estos valores en (3)
\begin{eqnarray}
    3{x}_{1} - {x}_{2} + {x}_{3} &=& 0 \nonumber \\
    3{x}_{1} - (-5{x}_{1}) + (-2{x}_{1}) &=& 0 \nonumber \\
    6{x}_{1} &=& 0 \nonumber \\
    {x}_{1} &=& 0 \nonumber
\end{eqnarray}
Reemplazando ${x}_{1} = 0$ en (2) se tiene que ${x}_{3} = 0$. Ahora por último si reemplazamos estos valores en (1) o (3) se tiene que ${x}_{2} = 0$. Por lo que el vector $\mathbf{x}$ estaría expresado de la siguiente forma
$$
\mathbf{x} = (0, 0, 0)
$$
\textbf{CONCLUSIÓN:} $A$ es linealmente independiente

\subsubsection{Para b)}
Tenemos los siguientes vectores
\begin{itemize}
    \item ${v}_{1} = (1,2,3)$
    \item ${v}_{2} = (4,5,6)$
    \item ${v}_{3} = (7,8,9)$ 
\end{itemize}
y el conjunto $B=\{ {v}_{1}, {v}_{2}, {v}_{3} \}$ expresando el conjunto en una matriz quedaría de la siguiente forma

\[
B = [ {v}_{1} \quad {v}_{2} \quad {v}_{3} ] =
\begin{bmatrix}
1 & 4 & 7 \\ 
2 & 5 & 8 \\
3 & 6 & 9
\end{bmatrix}
\Longrightarrow N(B) = \{ \mathbf{0} \} ?
\]
$$
N(B) = \{ \mathbf{x} \in \mathbb{R}^3 / B \mathbf{x} = \mathbf{0} \}
$$
Reduzcamos la matriz escalonada, mediante el programa nos da la siguiente matriz:
$$
\begin{bmatrix}
1 & 0 & -1 \\ 
0 & 1 & 2 \\
0 & 0 & 0
\end{bmatrix}
$$
La solución del sistema lineal estaría representado de la siguiente manera:
\[
\begin{bmatrix}
1 & 0 & -1 \\ 
0 & 1 & 2 \\
0 & 0 & 0
\end{bmatrix}
\begin{bmatrix}
{x}_{1} \\ 
{x}_{2} \\
{x}_{3}
\end{bmatrix} =
\begin{bmatrix}
0 \\ 
0 \\
0
\end{bmatrix}
\]
El cual nos da el siguiente sistema de combinaciones lineales
\setcounter{equation}{0}
\begin{eqnarray}
    {x}_{1} - {x}_{3} &=& 0 \\
    {x}_{2} + 2{x}_{3} &=& 0
\end{eqnarray}
De lo cual en la ecuación (1) se tiene que ${x}_{1} = {x}_{3}$, de la misma forma se tiene mediante la ecuación (2) que ${x}_{2} = -2{x}_{3}$ y si reemplazamos ${x}_{3}$ por ${x}_{1}$ en la ecuación (2) se tiene que ${x}_{2} = -2{x}_{1}$.\\

Expresando el vector $\mathbf{x}$ en función de ${x}_{1}$ se tiene lo siguiente
$$
\mathbf{x} = ({x}_{1}, {x}_{2}, {x}_{3}) = ({x}_{1}, -2{x}_{1}, {x}_{1}) = {x}_{1}(1, -2, 1)
$$
\textbf{CONCLUSIÓN:} $B$ no es linealmente independiente

\subsubsection{Para c)}
Tenemos los siguientes vectores
\begin{itemize}
    \item ${v}_{1} = (1,2,3,4)$
    \item ${v}_{2} = (1,1,1,1)$
    \item ${v}_{3} = (1,3,1,2)$
    \item ${v}_{4} = (2,1,1,2)$ 
\end{itemize}
y el conjunto $C=\{ {v}_{1}, {v}_{2}, {v}_{3}, {v}_{4} \}$ expresando el conjunto en una matriz quedaría de la siguiente forma

\[
C = [ {v}_{1} \quad {v}_{2} \quad {v}_{3} \quad {v}_{4} ] =
\begin{bmatrix}
1 & 1 & 1 & 2 \\ 
2 & 1 & 3 & 1 \\
3 & 1 & 1 & 1 \\
4 & 1 & 2 & 2 \\
\end{bmatrix}
\Longrightarrow N(C) = \{ \mathbf{0} \} ?
\]
$$
N(C) = \{ \mathbf{x} \in \mathbb{R}^4 / C \mathbf{x} = \mathbf{0} \}
$$
Reduzcamos la matriz escalonada, mediante el programa nos da la siguiente matriz:
$$
\begin{bmatrix}
1 & 0 & 0 & 0 \\ 
0 & 1 & 0 & 0 \\
0 & 0 & 1 & 0 \\
0 & 0 & 0 & 1
\end{bmatrix}
$$
La solución del sistema lineal estaría representado de la siguiente manera:
\[
\begin{bmatrix}
1 & 0 & 0 & 0 \\ 
0 & 1 & 0 & 0 \\
0 & 0 & 1 & 0 \\
0 & 0 & 0 & 1
\end{bmatrix}
\begin{bmatrix}
{x}_{1} \\ 
{x}_{2} \\
{x}_{3} \\
{x}_{4}
\end{bmatrix} =
\begin{bmatrix}
0 \\ 
0 \\
0 \\
0
\end{bmatrix}
\]
El cual nos da el siguiente sistema de combinaciones lineales
\setcounter{equation}{0}
\begin{eqnarray}
    {x}_{1} &=& 0 \\
    {x}_{2} &=& 0 \\
    {x}_{3} &=& 0 \\
    {x}_{4} &=& 0
\end{eqnarray}
Expresando el vector $\mathbf{x}$ de la siguiente manera
$$
\mathbf{x} = ({x}_{1}, {x}_{2}, {x}_{3}, {x}_{4}) = (0, 0, 0, 0)
$$
\textbf{CONCLUSIÓN:} $C$ es linealmente independiente

\subsubsection{Para d)}
Tenemos los siguientes vectores
\begin{itemize}
    \item ${v}_{1} = (1,2,-1,0)$
    \item ${v}_{2} = (0,1,2,1)$
    \item ${v}_{3} = (2,3,-4,-1)$ 
\end{itemize}
y el conjunto $D=\{ {v}_{1}, {v}_{2}, {v}_{3} \}$ expresando el conjunto en una matriz quedaría de la siguiente forma

\[
D = [ {v}_{1} \quad {v}_{2} \quad {v}_{3} ] =
\begin{bmatrix}
1 & 0 & 2 \\ 
2 & 1 & 3 \\
-1 & 2 & -4 \\
0 & 1 & -1
\end{bmatrix}
\Longrightarrow N(D) = \{ \mathbf{0} \} ?
\]
$$
N(D) = \{ \mathbf{x} \subset \mathbb{R}^4 / D \mathbf{x} = \mathbf{0} \}
$$
Reduzcamos la matriz escalonada, mediante el programa nos da la siguiente matriz:
$$
\begin{bmatrix}
1 & 0 & 2 \\ 
0 & 1 & -1 \\
0 & 0 & 0 \\
0 & 0 & 0
\end{bmatrix}
$$
La solución del sistema lineal estaría representado de la siguiente manera:
\[
\begin{bmatrix}
1 & 0 & 2 \\ 
0 & 1 & -1 \\
0 & 0 & 0 \\
0 & 0 & 0
\end{bmatrix}
\begin{bmatrix}
{x}_{1} \\ 
{x}_{2} \\
{x}_{3}
\end{bmatrix} =
\begin{bmatrix}
0 \\ 
0 \\
0 \\
0
\end{bmatrix}
\]
El cual nos da el siguiente sistema de combinaciones lineales
\setcounter{equation}{0}
\begin{eqnarray}
    {x}_{1} + 2{x}_{3} &=& 0 \\
    {x}_{2} - {x}_{3} &=& 0
\end{eqnarray}
De lo cual en la expresión (2) se tiene que ${x}_{2} = {x}_{3}$, en la expresión (1) se tiene que ${x}_{1} = -2 {x}_{3}$. Asimismo reemplazando ${x}_{3}$ por $-\frac{1}{2}{x}_{1}$ en la ecuación (2) se tiene que ${x}_{2} = -\frac{1}{2}{x}_{1}$

Expresando el vector $\mathbf{x}$ en función de ${x}_{1}$ se queda de la siguiente manera
$$
\mathbf{x} = ({x}_{1}, {x}_{2}, {x}_{3}) = \left( {x}_{1}, -\frac{1}{2}{x}_{1}, -\frac{1}{2}{x}_{1} \right) = {x}_{1} \left(1, -\frac{1}{2}, -\frac{1}{2} \right)
$$
\textbf{CONCLUSIÓN:} $D$ no es linealmente independiente

\section{Problema}

Sea $A$ una matriz $mxn$, y sea $X \subseteq \mathbb{R}^n$ un subconjunto arbitrario de $\mathbb{R}^n$. La imagen de $X$ bajo la transformación de $A$ se define como el conjunto

$$
A(X) = \{ A \mathbf{x} : \mathbf{x} \in X \}
$$

\begin{itemize}
    \item[(a)] Demuestre que $A(X) \subseteq C(A)$

    \item[(b)] Si $X$ es un subespacio de $\mathbb{R}^n$, pruebe que $A(X)$ es un subespacio de $\mathbb{R}^m$.
\end{itemize}

\subsection{Solucion a)}
Utilizando la definición de rango de una matriz definimos $C(A)$
\begin{defi}\textbf{Rango de matriz}
    \\
    El rango de una matriz $A$ es la dimensión de $C(A)$ donde
    $$
    C(A) = \{ \mathbf{y} \in \mathbb{R}^m / A\mathbf{x} = \mathbf{y}, \text{para algún }\mathbf{x} \in \mathbb{R}^n \}
    $$
\end{defi}

Tomemos en cuenta que 
$$
A(X) = \{ A \mathbf{x} / \mathbf{x} \in X \subseteq \mathbb{R}^n \}
$$

Es decir que cualquier $\mathbf{x}$ que pertenece a $X$, pertenece a $\mathbb{R}^n$ \\

Ahora de $C(A)$ se tiene
$$
C(A) = \{ \mathbf{y} \in \mathbb{R}^m / A \mathbf{x} = \mathbf{y}, \text{para algún }\mathbf{x} \in \mathbb{R}^n \}
$$
Cada $A\mathbf{x} \in A(X)$ también está incluido en $C(A)$ por lo que generalizando se tiene que
$$
A(X) \subseteq C(A)
$$

\subsection{Solucion b)}
Para probar tenemos que comprobar la misma definición que se planteo en el problema 1 acerca de espacio vectorial en $\mathbb{R}^n$

\subsubsection{Primera condición}
Sean los vectores $\mathbf{u}$, $\mathbf{v} \in A(X)$ de donde podemos expresarlos de la siguiente manera:
$$
\mathbf{u} = Ax \quad \text{y} \quad \mathbf{v} = Ay
$$
Lo cual indica que $x,y \in X$ entonces tenemos que demostrar que $\mathbf{u}+\mathbf{v} \in A(X)$, desarrollando esto tenemos
$$
\mathbf{u} + \mathbf{v} = Ax + Ay = A(x+y)
$$
De lo cual se sabe que como $x, y \in X$, también $(x+y) \in X$, se observa que $A(x+y)$ esta incluido en el conjunto, es decir $\mathbf{u} + \mathbf{v} \in A(X)$

\subsubsection{Segunda condición}
Sea $c \in \mathbb{R}$ un escalar, debemos demotrar que $c\mathbf{u} \in A(X)$, desarrollando se tiene la siguiente expresión
$$
c \mathbf{u} = c (Ax) = A(cx)
$$
De lo cual se sabe que como $x \in X$, también $cx \in X$, para todo $c \in \mathbb{R}$ se observa que $A(cx)$ esta incluido en el conjunto, es decir $c\mathbf{u} \in A(X)$\\

\textbf{CONCLUSIÓN:} Si $X$ es un subespacio de $\mathbb{R}^n, A(X)$ es un subespacio de $\mathbb{R}^{m}$

\section{Problema}

Considere los siguientes vectores en $\mathbb{R}^4$:

\[
    \left\{ 
    \begin{bmatrix}
    1 \\ 
    1 \\ 
    1 \\
    1
    \end{bmatrix}
    ,
    \begin{bmatrix}
    0 \\ 
    1 \\ 
    1 \\
    1
    \end{bmatrix}
    ,
    \begin{bmatrix}
    0 \\ 
    0 \\ 
    1 \\
    1
    \end{bmatrix}
    ,
    \begin{bmatrix}
    0 \\ 
    0 \\ 
    0 \\
    1
    \end{bmatrix} \right\}.
    \]

Aplique el procedimiento de Gram-Schmidt a estos vectores para obtener una base ortonormal de $\mathbb{R}^4$.

\subsection{Solución}
Para obtener la base ortonormal utilizamos la definición de Ortogonalización de Gram-Schmidt

\begin{defi}\textbf{Ortogonalización de Gram-Schmidt}
    \\
    Sea $A = \{ {v}_{1}, \cdots , {v}_{n} \}$ una base para un espacio euclidiano
    \begin{itemize}
        \item ${w}_{1} = {v}_{1} \quad ; \quad {u}_{1} = \frac{{w}_{1}}{\|{w}_{1} \|}$
        \item ${w}_{2} = {v}_{2} - {\textsl{proy}}_{{w}_{1}}({v}_{2}) \quad ; \quad {u}_{2} = \frac{{w}_{2}}{\|{w}_{2} \|}$
        \item ${w}_{3} = {v}_{3} - {\textsl{proy}}_{{w}_{1}}({v}_{2}) - {\textsl{proy}}_{{w}_{2}}({v}_{3}) \quad ; \quad {u}_{3} = \frac{{w}_{3}}{\|{w}_{3} \|}$
        $$
        \vdots 
        $$
        \item ${w}_{n} = {v}_{n} - {\textsl{proy}}_{{w}_{1}}({v}_{2}) - \cdots - {\textsl{proy}}_{{w}_{n-1}}({v}_{n}) \quad ; \quad {u}_{n} = \frac{{w}_{n}}{\|{w}_{n} \|}$
        \item $\{ {w}_{1}, \cdots , {w}_{n} \}$ es ortogonal.
        \item $\{ {u}_{1}, \cdots , {u}_{n} \}$ es ortonormal.
    \end{itemize}
\end{defi}

En nuestro caso tenemos los vectores en $\mathbb{R}^4$ de la siguiente manera $A = \{ {v}_{1}, {v}_{2}, {v}_{3}, {v}_{4} \}$ donde los vectores estan expresados de la siguiente manera
\begin{itemize}
    \item ${v}_{1} = (1, 1, 1, 1)$
    \item ${v}_{2} = (0, 1, 1, 1)$
    \item ${v}_{3} = (0, 0, 1, 1)$
    \item ${v}_{4} = (0, 0, 0, 1)$
\end{itemize}
Tenemos los siguientes ejemplos hasta $n = 4$
\begin{itemize}
    \item ${w}_{1} = {v}_{1} \quad ; \quad {u}_{1} = \frac{{w}_{1}}{\|{w}_{1} \|}$
    \item ${w}_{2} = {v}_{2} - {\textsl{proy}}_{{w}_{1}}({v}_{2}) \quad ; \quad {u}_{2} = \frac{{w}_{2}}{\|{w}_{2} \|}$
    \item ${w}_{3} = {v}_{3} - {\textsl{proy}}_{{w}_{1}}({v}_{2}) - {\textsl{proy}}_{{w}_{2}}({v}_{3}) \quad ; \quad {u}_{3} = \frac{{w}_{3}}{\|{w}_{3} \|}$
    \item ${w}_{4} = {v}_{4} - {\textsl{proy}}_{{w}_{1}}({v}_{2}) - {\textsl{proy}}_{{w}_{2}}({v}_{3}) - {\textsl{proy}}_{{w}_{3}}({v}_{4}) \quad ; \quad {u}_{4} = \frac{{w}_{4}}{\|{w}_{4} \|}$
    \item $\{ {w}_{1}, {w}_{2}, {w}_{3} , {w}_{4} \}$ es ortogonal.
    \item $\{ {u}_{1}, {u}_{2} , {u}_{3}, {u}_{4} \}$ es ortonormal.
\end{itemize}
En lo cual debemos hallar los vectores ortonormales ${u}_{i}$, desarrollemos de la siguiente manera
\begin{enumerate}
    \item \textbf{Hallar el primer vector ortonormal}\\
    Como ${w}_{1} = {v}_{1}$, reemplazamos en la ecuación
    \begin{eqnarray}
        {u}_{1} &=& \frac{{w}_{1}}{\|{w}_{1} \|} \nonumber \\
        {u}_{1} &=& \frac{{v}_{1}}{\|{v}_{1} \|} \nonumber \\
        {u}_{1} &=& \frac{(1,1,1,1)}{\sqrt{\langle {v}_{1}, {v}_{1} \rangle}} \nonumber \\
        {u}_{1} &=& \frac{(1,1,1,1)}{\sqrt{{1}^{2}+{1}^{2}+{1}^{2}+{1}^{2}}} \nonumber \\
        {u}_{1} &=& \frac{(1,1,1,1)}{\sqrt{4}} \nonumber \\
        {u}_{1} &=& \frac{(1,1,1,1)}{2} \nonumber \\
        {u}_{1} &=& \left( \frac{1}{2}, \frac{1}{2}, \frac{1}{2}, \frac{1}{2} \right) \nonumber
    \end{eqnarray}
    \item \textbf{Hallar el segundo vector ortonormal}\\
    En este caso primero tenemos que calcular ${w}_{2}$ el cual sería de la siguiente manera
    \begin{eqnarray}
        {w}_{2} &=& {v}_{2} - {\textsl{proy}}_{{w}_{1}}({v}_{2}) \nonumber \\
        {w}_{2} &=& (0,1,1,1) - \frac{\langle {w}_{1}, {v}_{2} \rangle}{{\| {w}_{1} \|}^{2}} {w}_{1} \nonumber \\
        {w}_{2} &=& (0,1,1,1) - \frac{\langle (1,1,1,1), (0,1,1,1) \rangle}{{2}^{2}} (1,1,1,1) \nonumber \\
        {w}_{2} &=& (0,1,1,1) - \frac{0+1+1+1}{4} (1,1,1,1) \nonumber \\
        {w}_{2} &=& (0,1,1,1) - \frac{3}{4} (1,1,1,1) \nonumber \\
        {w}_{2} &=& (0,1,1,1) - \left( \frac{3}{4}, \frac{3}{4}, \frac{3}{4}, \frac{3}{4} \right)  \nonumber \\
        {w}_{2} &=& \left(- \frac{3}{4}, \frac{1}{4}, \frac{1}{4}, \frac{1}{4} \right) \nonumber
    \end{eqnarray}
    Ahora calculemos el segundo vector ortonormal ${u}_{2}$
    \begin{eqnarray}
        {u}_{2} &=& \frac{{w}_{2}}{\|{w}_{2} \|} \nonumber \\
        {u}_{2} &=& \frac{\left(- \frac{3}{4}, \frac{1}{4}, \frac{1}{4}, \frac{1}{4} \right)}{\sqrt{\langle {w}_{2}, {w}_{2} \rangle}} \nonumber \\
        {u}_{2} &=& \frac{\left(- \frac{3}{4}, \frac{1}{4}, \frac{1}{4}, \frac{1}{4} \right)}{\sqrt{{\left(-\frac{3}{4} \right)}^{2}+{\left(\frac{1}{4} \right)}^{2}+{\left(\frac{1}{4} \right)}^{2}+{\left(\frac{1}{4} \right)}^{2}}} \nonumber \\
        {u}_{2} &=& \frac{\left(- \frac{3}{4}, \frac{1}{4}, \frac{1}{4}, \frac{1}{4} \right)}{\sqrt{\left(\frac{9}{16}\right) + \left(\frac{1}{16}\right) + \left(\frac{1}{16}\right) + \left(\frac{1}{16}\right) }} \nonumber \\
        {u}_{2} &=& \frac{\left(- \frac{3}{4}, \frac{1}{4}, \frac{1}{4}, \frac{1}{4} \right)}{\sqrt{\left( \frac{12}{16} \right) }} \nonumber \\
        {u}_{2} &=& \frac{\left(- \frac{3}{4}, \frac{1}{4}, \frac{1}{4}, \frac{1}{4} \right)}{\frac{2\sqrt{3}}{4}} \nonumber \\
        {u}_{2} &=& \left(- \frac{3}{2 \sqrt{3}}, \frac{1}{2 \sqrt{3}}, \frac{1}{2 \sqrt{3}}, \frac{1}{2 \sqrt{3}} \right) \nonumber
    \end{eqnarray}
    \item \textbf{Hallar el tercer vector ortonormal}\\
    En este caso primero tenemos que calcular ${w}_{3}$ el cual sería de la siguiente manera
    \begin{eqnarray}
        {w}_{3} &=& {v}_{3} - {\textsl{proy}}_{{w}_{1}}({v}_{2}) - {\textsl{proy}}_{{w}_{2}}({v}_{3}) \nonumber \\
        {w}_{3} &=& (0,0,1,1) - \frac{\langle {w}_{1}, {v}_{2} \rangle}{{\| {w}_{1} \|}^{2}} {w}_{1} - \frac{\langle {w}_{2}, {v}_{3} \rangle}{{\| {w}_{2} \|}^{2}} {w}_{2} \nonumber \\
        {w}_{3} &=& (0,0,1,1) - \left( \frac{3}{4}, \frac{3}{4}, \frac{3}{4}, \frac{3}{4} \right) - \frac{\langle \left(- \frac{3}{4}, \frac{1}{4}, \frac{1}{4}, \frac{1}{4} \right), (0,0,1,1) \rangle}{{\left( \frac{2\sqrt{3}}{4} \right)}^{2}} \left(- \frac{3}{4}, \frac{1}{4}, \frac{1}{4}, \frac{1}{4} \right) \nonumber \\
        {w}_{3} &=& \left(- \frac{3}{4},- \frac{3}{4}, \frac{1}{4}, \frac{1}{4} \right) - \frac{\frac{1}{2}}{\frac{3}{4}} \left(- \frac{3}{4}, \frac{1}{4}, \frac{1}{4}, \frac{1}{4} \right) \nonumber \\
        {w}_{3} &=& \left(- \frac{3}{4},- \frac{3}{4}, \frac{1}{4}, \frac{1}{4} \right) - \frac{2}{3} \left(- \frac{3}{4}, \frac{1}{4}, \frac{1}{4}, \frac{1}{4} \right) \nonumber \\
        {w}_{3} &=& \left(- \frac{3}{4},- \frac{3}{4}, \frac{1}{4}, \frac{1}{4} \right) - \left(- \frac{1}{2}, \frac{1}{6}, \frac{1}{6}, \frac{1}{6} \right)  \nonumber \\
        {w}_{3} &=& \left(- \frac{1}{4},- \frac{11}{12}, \frac{1}{12}, \frac{1}{12} \right) \nonumber
    \end{eqnarray}
    Ahora calculemos el tercer vector ortonormal ${u}_{3}$
    \begin{eqnarray}
        {u}_{3} &=& \frac{{w}_{3}}{\|{w}_{3} \|} \nonumber \\
        {u}_{3} &=& \frac{\left(- \frac{1}{4},- \frac{11}{12}, \frac{1}{12}, \frac{1}{12} \right)}{\sqrt{\langle {w}_{3}, {w}_{3} \rangle}} \nonumber \\
        {u}_{3} &=& \frac{\left(- \frac{1}{4},- \frac{11}{12}, \frac{1}{12}, \frac{1}{12} \right)}{\sqrt{{\left(-\frac{1}{4} \right)}^{2}+{\left(-\frac{11}{12} \right)}^{2}+{\left(\frac{1}{12} \right)}^{2}+{\left(\frac{1}{12} \right)}^{2}}} \nonumber \\
        {u}_{3} &=& \frac{\left(- \frac{1}{4},- \frac{11}{12}, \frac{1}{12}, \frac{1}{12} \right)}{\sqrt{\left(\frac{1}{16}\right) + \left(\frac{121}{144}\right) + \left(\frac{1}{144}\right) + \left(\frac{1}{144}\right) }} \nonumber \\
        {u}_{3} &=& \frac{\left(- \frac{1}{4},- \frac{11}{12}, \frac{1}{12}, \frac{1}{12} \right)}{\sqrt{\left( \frac{132}{144} \right) }} \nonumber \\
        {u}_{3} &=& \frac{\left(- \frac{1}{4},- \frac{11}{12}, \frac{1}{12}, \frac{1}{12} \right)}{\frac{2\sqrt{33}}{12}} \nonumber \\
        {u}_{3} &=& \left(- \frac{3}{2 \sqrt{33}},- \frac{11}{2 \sqrt{33}}, \frac{1}{2 \sqrt{33}}, \frac{1}{2 \sqrt{33}} \right) \nonumber
    \end{eqnarray}
    \item \textbf{Hallar el cuarto vector ortonormal}\\
    En este caso primero tenemos que calcular ${w}_{4}$ el cual sería de la siguiente manera
    \begin{eqnarray}
        {w}_{4} &=& {v}_{4} - {\textsl{proy}}_{{w}_{1}}({v}_{2}) - {\textsl{proy}}_{{w}_{2}}({v}_{3}) - {\textsl{proy}}_{{w}_{3}}({v}_{4}) \nonumber \\
        {w}_{4} &=& (0,0,0,1) - \frac{\langle {w}_{1}, {v}_{2} \rangle}{{\| {w}_{1} \|}^{2}} {w}_{1} - \frac{\langle {w}_{2}, {v}_{3} \rangle}{{\| {w}_{2} \|}^{2}} {w}_{2} - \frac{\langle {w}_{3}, {v}_{4} \rangle}{{\| {w}_{3} \|}^{2}} {w}_{3} \nonumber \\
        {w}_{4} &=& (0,0,0,1) - \left( \frac{3}{4}, \frac{3}{4}, \frac{3}{4}, \frac{3}{4} \right) - \left(- \frac{1}{2}, \frac{1}{6}, \frac{1}{6}, \frac{1}{6} \right) - \frac{\langle \left(- \frac{1}{4},- \frac{11}{12}, \frac{1}{12}, \frac{1}{12} \right), (0,0,0,1) \rangle}{{\left( \frac{2\sqrt{33}}{12} \right)}^{2}} \left(- \frac{1}{4},- \frac{11}{12}, \frac{1}{12}, \frac{1}{12} \right) \nonumber \\
        {w}_{4} &=& \left(- \frac{1}{4},- \frac{11}{12},- \frac{11}{12}, \frac{1}{12} \right) - \frac{\frac{1}{12}}{\frac{11}{12}} \left(- \frac{1}{4},- \frac{11}{12}, \frac{1}{12}, \frac{1}{12} \right) \nonumber \\
        {w}_{4} &=& \left(- \frac{1}{4},- \frac{11}{12},- \frac{11}{12}, \frac{1}{12} \right) - \frac{1}{11} \left(- \frac{1}{4},- \frac{11}{12}, \frac{1}{12}, \frac{1}{12} \right) \nonumber \\
        {w}_{4} &=& \left(- \frac{1}{4},- \frac{11}{12},- \frac{11}{12}, \frac{1}{12} \right) - \left(- \frac{1}{44},- \frac{1}{12}, \frac{1}{132}, \frac{1}{132} \right)  \nonumber \\
        {w}_{4} &=& \left(- \frac{5}{22},- \frac{5}{6},- \frac{61}{66}, \frac{5}{66} \right) \nonumber
    \end{eqnarray}
    Ahora calculemos el cuarto vector ortonormal ${u}_{4}$
    \begin{eqnarray}
        {u}_{4} &=& \frac{{w}_{4}}{\|{w}_{4} \|} \nonumber \\
        {u}_{4} &=& \frac{\left(- \frac{5}{22},- \frac{5}{6},- \frac{61}{66}, \frac{5}{66} \right)}{\sqrt{\langle {w}_{4}, {w}_{4} \rangle}} \nonumber \\
        {u}_{4} &=& \frac{\left(- \frac{5}{22},- \frac{5}{6},- \frac{61}{66}, \frac{5}{66} \right)}{\sqrt{{\left(-\frac{5}{22} \right)}^{2}+{\left(-\frac{5}{6} \right)}^{2}+{\left(-\frac{61}{66} \right)}^{2}+{\left(\frac{5}{66} \right)}^{2}}} \nonumber \\
        {u}_{4} &=& \frac{\left(- \frac{5}{22},- \frac{5}{6},- \frac{61}{66}, \frac{5}{66} \right)}{\sqrt{\left(\frac{25}{484}\right) + \left(\frac{25}{36}\right) + \left(\frac{3721}{4356}\right) + \left(\frac{25}{4356}\right) }} \nonumber \\
        {u}_{4} &=& \frac{\left(- \frac{5}{22},- \frac{5}{6},- \frac{61}{66}, \frac{5}{66} \right)}{\sqrt{\left( \frac{6996}{4356} \right) }} \nonumber \\
        {u}_{4} &=& \frac{\left(- \frac{5}{22},- \frac{5}{6},- \frac{61}{66}, \frac{5}{66} \right)}{\frac{\sqrt{53}}{\sqrt{33}}} \nonumber \\
        {u}_{4} &=& \left(- \frac{5 \sqrt{33}}{22 \sqrt{53}},- \frac{5 \sqrt{33}}{6 \sqrt{53}},- \frac{61 \sqrt{33}}{66 \sqrt{53}}, \frac{5 \sqrt{33}}{66 \sqrt{53}} \right) \nonumber
    \end{eqnarray}
\end{enumerate}

\textbf{CONCLUSIÓN:} La base ortonormal está formado por los vectores $u = \{ {u}_{1}, {u}_{2} , {u}_{3}, {u}_{4} \}$ expresados de la siguiente manera:

\[
u = 
    \left\{ 
    \begin{bmatrix}
    \frac{1}{2} \\ 
    \frac{1}{2} \\ 
    \frac{1}{2} \\
    \frac{1}{2}
    \end{bmatrix}
    ,
    \begin{bmatrix}
    -\frac{3}{2 \sqrt{3}} \\ 
    \frac{1}{2 \sqrt{3}} \\ 
    \frac{1}{2 \sqrt{3}} \\
    \frac{1}{2 \sqrt{3}}
    \end{bmatrix}
    ,
    \begin{bmatrix}
    -\frac{3}{2 \sqrt{33}} \\ 
    -\frac{11}{2 \sqrt{33}} \\ 
    \frac{1}{2 \sqrt{33}} \\
    \frac{1}{2 \sqrt{33}}
    \end{bmatrix}
    ,
    \begin{bmatrix}
    -\frac{5 \sqrt{33}}{22 \sqrt{53}} \\ 
    -\frac{5 \sqrt{33}}{6 \sqrt{53}} \\ 
    -\frac{61 \sqrt{33}}{66 \sqrt{53}} \\
    \frac{5 \sqrt{33}}{66 \sqrt{53}}
    \end{bmatrix} \right\}.
    \]