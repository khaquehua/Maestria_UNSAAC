
%\tableofcontents
\newpage

%%%%%%%%%%%%%%%%%%%%%%
%%%%%%%%% Capítulo I
%%%%%%%%%%%%%%%%%%%%%%%

\section{Problema}

Demuestre que $S$ es un subespacio de $\mathbb{R}^3$, donde

\[
S = \left\{ \mathbf{x} \in \mathbb{R}^3 : \mathbf{x} = \alpha
\begin{bmatrix}
1 \\ 
2 \\ 
3
\end{bmatrix}
+ \beta
\begin{bmatrix}
2 \\ 
1 \\ 
0
\end{bmatrix}, \, \alpha, \beta \in \mathbb{R} \right\}.
\]

\subsection{Solución}
Para demostrar que $S$ es un subespacio vectorial de $\mathbb{R}^3$, comprobemos que cumpla la siguiente definición

\begin{defi}\textbf{Espacio vectorial en $\mathbb{R}^n$}
    \\
    Un conjunto $S$ no vacío de $\mathbb{R}^n$, es llamado un espacio vectorial en $\mathbb{R}^n$ si verifica las siguientes condiciones
    \begin{itemize}
        \item[i)] Para $\mathbf{x},\mathbf{y} \in S$, entonces $\mathbf{x}+\mathbf{y} \in S$ 
        \item[ii)] Para $\alpha \in \mathbb{R}$, $\mathbf{x} \in S$, entonces $\alpha \mathbf{x} \in S$ 
    \end{itemize}
\end{defi}

\subsubsection{Primera condición}
Sean $\mathbf{x}$,$\mathbf{y} \in S$, expresados de la siguiente manera:

\[
\mathbf{x} = {\alpha}_{1} 
\begin{bmatrix}
1 \\ 
2 \\ 
3
\end{bmatrix}
+
{\beta}_{1}
\begin{bmatrix}
2 \\ 
1 \\ 
0
\end{bmatrix},
\qquad
\mathbf{y} = {\alpha}_{2} 
\begin{bmatrix}
1 \\ 
2 \\ 
3
\end{bmatrix}
+
{\beta}_{2}
\begin{bmatrix}
2 \\ 
1 \\ 
0
\end{bmatrix},
\]

Donde ${\alpha}_{1},{\alpha}_{2},{\beta}_{1},{\beta}_{2} \in \mathbb{R}$, realizemos la siguiente operación $\mathbf{x}+\mathbf{y}$ el cual debe pertenecer a $S$

\[
\mathbf{x} + \mathbf{y} = \left( {\alpha}_{1} 
\begin{bmatrix}
1 \\ 
2 \\ 
3
\end{bmatrix}
+
{\beta}_{1}
\begin{bmatrix}
2 \\ 
1 \\ 
0
\end{bmatrix} \right)
+
\left( {\alpha}_{2} 
\begin{bmatrix}
1 \\ 
2 \\ 
3
\end{bmatrix}
+
{\beta}_{2}
\begin{bmatrix}
2 \\ 
1 \\ 
0
\end{bmatrix} \right)
\]

En esta parte se puede agrupar los vectores $\begin{bmatrix}
1 \\ 
2 \\ 
3
\end{bmatrix}$ y
$
\begin{bmatrix}
2 \\ 
1 \\ 
0
\end{bmatrix}$ como términos comúnes

\[
\mathbf{x} + \mathbf{y} = ({\alpha}_{1}+{\alpha}_{2}) 
\begin{bmatrix}
1 \\ 
2 \\ 
3
\end{bmatrix}
+
({\beta}_{1}+{\beta}_{2})
\begin{bmatrix}
2 \\ 
1 \\ 
0
\end{bmatrix}
\]

Se sabe que si ${\alpha}_{1},{\alpha}_{2},{\beta}_{1},{\beta}_{2} \in \mathbb{R}$ entonces ${\alpha}_{1}+{\alpha}_{2} \in \mathbb{R}$, de igual forma ${\beta}_{1}+{\beta}_{2} \in \mathbb{R}$, entonces se cumple que $\mathbf{x} +\mathbf{y} \in S$

\subsubsection{Segunda condición}
Sea $\mathbf{x} \in S$, esto es

\[
\mathbf{x} = \alpha 
\begin{bmatrix}
1 \\ 
2 \\ 
3
\end{bmatrix}
+
\beta
\begin{bmatrix}
2 \\ 
1 \\ 
0
\end{bmatrix}, \alpha , \beta \in \mathbb{R}
\]

Multipliquemos por un escalar $\gamma \in \mathbb{R}$:

\[
\gamma \mathbf{x} = \gamma \left( \alpha 
\begin{bmatrix}
1 \\ 
2 \\ 
3
\end{bmatrix}
+
\beta
\begin{bmatrix}
2 \\ 
1 \\ 
0
\end{bmatrix} \right)
\]

\[
\gamma \mathbf{x} = \gamma \alpha 
\begin{bmatrix}
1 \\ 
2 \\ 
3
\end{bmatrix}
+
\gamma \beta
\begin{bmatrix}
2 \\ 
1 \\ 
0
\end{bmatrix}
\]

Se sabe que si ${\alpha},{\beta},{\gamma} \in \mathbb{R}$ entonces ${\alpha}{\gamma} \in \mathbb{R}$, de igual forma ${\beta}{\gamma} \in \mathbb{R}$, entonces se cumple que $\gamma \mathbf{x} \in S$ \\

\textbf{CONCLUSIÓN: }$S$ es un subespacio vectorial de $\mathbb{R}^3$

\section{Problema}

Considere el siguiente conjunto de vectores en $\mathbb{R}^3$:

\[
X = \left\{ 
\begin{bmatrix}
1 \\ 
0 \\ 
-1
\end{bmatrix}
,
\begin{bmatrix}
-2 \\ 
1 \\ 
1
\end{bmatrix} \right\}.
\]

Demuestre que uno de los dos vectores

\[
\mathbf{u} = 
\begin{bmatrix}
-1 \\ 
2 \\ 
1
\end{bmatrix}
\quad
\text{y}
\quad
\mathbf{v} = 
\begin{bmatrix}
-1 \\ 
1 \\ 
1
\end{bmatrix}
\]

pertenece a $gen(X)$, mientras que el otro no.\\
Encuentre un número real de $\lambda$ tal que

\[
\begin{bmatrix}
1 \\ 
1 \\ 
\lambda
\end{bmatrix}
\in gen(X)
\]

\subsection{Solución}
Para demostrar que uno de los $\mathbf{u},\mathbf{v} \in gen(X)$ se debe cumplir la siguiente definición

\begin{defi}\textbf{Espacio generado por los vectores}
    \\
    El conjunto de todas las posibles combinaciones lineales de ${v}_{1} , {v}_{2} , \cdots , {v}_{r}$ es llamado espacio generado $gen(X)$ por ${v}_{1} , \cdots , {v}_{n}$ si
    $$
    gen({v}_{1} , \cdots , {v}_{n} ) = \{ {\alpha}_{1} {v}_{1} + \cdots + {\alpha}_{r} {v}_{r} / {\alpha}_{i} \in \mathbb{R} , i = 1 , \cdots , r \}
    $$
\end{defi}
\subsubsection{Para el vector $\mathbf{u}$}
Se debe cumplir la siguiente combinación lineal

\[
\mathbf{u} = {\alpha}_{1}
\begin{bmatrix}
1 \\ 
0 \\ 
-1
\end{bmatrix}
+ {\alpha}_{2}
\begin{bmatrix}
-2 \\ 
1 \\ 
1
\end{bmatrix}
\]

Reemplanzado por el vector $\mathbf{u}$ se tiene la siguiente igualdad

\[
\begin{bmatrix}
-1 \\ 
2 \\ 
1
\end{bmatrix}
= {\alpha}_{1}
\begin{bmatrix}
1 \\ 
0 \\ 
-1
\end{bmatrix}
+ {\alpha}_{2}
\begin{bmatrix}
-2 \\ 
1 \\ 
1
\end{bmatrix}
\]

Se tienen las siguientes combinaciones lineales
\begin{eqnarray}
    {\alpha}_{1} - 2 {\alpha}_{2} &=& -1 \\
    {\alpha}_{2} &=& 2 \\
    -{\alpha}_{1} + {\alpha}_{2} &=& 1
\end{eqnarray}
Del cual de (2) se tiene que $\alpha_2 = 2$, reemplazemos estos valores en (1) y (3):
\begin{itemize}
     \item Reemplanzado en (1):
     \begin{eqnarray}
         {\alpha}_{1} - 2 {\alpha}_{2} &=& -1 \nonumber \\
         {\alpha}_{1} - 2 (2) &=& -1 \nonumber \\
         {\alpha}_{1} &=& 3 \nonumber
     \end{eqnarray}
     Se observa que ${\alpha}_{1} = 3$, y cumple la igualdad en $\mathbf{u}$, ahora reemplazemos ${\alpha}_{1} = 3$ y ${\alpha}_{2} = 2$ en (3) 
     \item Reemplanzado en (3):
     \begin{eqnarray}
         -{\alpha}_{1} + {\alpha}_{2} &=& 1 \nonumber \\
         -(3) + (2) &=& -1 \neq 1 \nonumber
     \end{eqnarray}
     Se observa que no cumple la igualdad.
 \end{itemize} 
Por lo tanto $\mathbf{u} \notin gen(X)$

\subsubsection{Para el vector $\mathbf{v}$}
Se debe cumplir la siguiente combinación lineal

\[
\mathbf{v} = {\alpha}_{1}
\begin{bmatrix}
1 \\ 
0 \\ 
-1
\end{bmatrix}
+ {\alpha}_{2}
\begin{bmatrix}
-2 \\ 
1 \\ 
1
\end{bmatrix}
\]

Reemplanzado por el vector $\mathbf{v}$ se tiene la siguiente igualdad

\[
\begin{bmatrix}
-1 \\ 
1 \\ 
1
\end{bmatrix}
= {\alpha}_{1}
\begin{bmatrix}
1 \\ 
0 \\ 
-1
\end{bmatrix}
+ {\alpha}_{2}
\begin{bmatrix}
-2 \\ 
1 \\ 
1
\end{bmatrix}
\]
\setcounter{equation}{0}
Se tienen las siguientes combinaciones lineales
\begin{eqnarray}
    {\alpha}_{1} - 2 {\alpha}_{2} &=& -1 \\
    {\alpha}_{2} &=& 1 \\
    -{\alpha}_{1} + {\alpha}_{2} &=& 1
\end{eqnarray}
Del cual de (2) se tiene que $\alpha_2 = 1$, reemplazemos estos valores en (1) y (3):
\begin{itemize}
     \item Reemplanzado en (1):
     \begin{eqnarray}
         {\alpha}_{1} - 2 {\alpha}_{2} &=& -1 \nonumber \\
         {\alpha}_{1} - 2 (1) &=& -1 \nonumber \\
         {\alpha}_{1} &=& 1 \nonumber
     \end{eqnarray}
     Se observa que ${\alpha}_{1} = 1$, y cumple la igualdad en $\mathbf{v}$, ahora reemplazemos ${\alpha}_{1} = 1$ y ${\alpha}_{2} = 1$ en (3) 
     \item Reemplanzado en (3):
     \begin{eqnarray}
         -{\alpha}_{1} + {\alpha}_{2} &=& 1 \nonumber \\
         -(1) + (1) &=& 0 \neq 1 \nonumber
     \end{eqnarray}
     Se observa que no cumple la igualdad.\\
     Por lo tanto $\mathbf{v} \notin gen(X)$
 \end{itemize} 

\textbf{CONCLUSIÓN:} Ninguno de los vectores $\mathbf{u},\mathbf{v} \notin gen(X)$

\subsection{Encontrar un número real $\lambda$}
De igual forma igualamos la combinación lineal e igualamos al vector para hallar el valor de $\lambda$

\setcounter{equation}{0}
Se tienen las siguientes combinaciones lineales
\begin{eqnarray}
    {\alpha}_{1} - 2 {\alpha}_{2} &=& 1 \\
    {\alpha}_{2} &=& 1 \\
    -{\alpha}_{1} + {\alpha}_{2} &=& \lambda
\end{eqnarray}
Del cual de (2) se tiene que $\alpha_2 = 1$, reemplazemos estos valores en (1) y (3):
\begin{itemize}
     \item Reemplanzado en (1):
     \begin{eqnarray}
         {\alpha}_{1} - 2 {\alpha}_{2} &=& 1 \nonumber \\
         {\alpha}_{1} - 2 (1) &=& 1 \nonumber \\
         {\alpha}_{1} &=& 3 \nonumber
     \end{eqnarray}
     Se observa que ${\alpha}_{1} = 3$ y ${\alpha}_{2} = 1$, reemplazemos estos valores en (3) 
     \item Reemplanzado en (3):
     \begin{eqnarray}
         -{\alpha}_{1} + {\alpha}_{2} &=& \lambda \nonumber \\
         -(3) + (1) &=& \lambda \nonumber \\
         \lambda &=& -2 \nonumber
     \end{eqnarray}
     El valor de $\lambda$ para que el vector $\begin{bmatrix}
1 \\ 
1 \\ 
\lambda
\end{bmatrix} \in gen(X)$ es -2 
 \end{itemize} 

\textbf{CONCLUSIÓN:} El número real de $\lambda$ es -2

\section{Problema}

Verifique si cada uno de los siguientes conjuntos de vectores son linealmente independientes:

\begin{enumerate}[label=\alph*)] % Cambia el formato a a), b), c), etc.
    \item 

    \[
    \left\{ 
    \begin{bmatrix}
    1 \\ 
    2 \\ 
    3
    \end{bmatrix}
    ,
    \begin{bmatrix}
    1 \\ 
    0 \\ 
    -1
    \end{bmatrix}
    ,
    \begin{bmatrix}
    -2 \\ 
    1 \\ 
    1
    \end{bmatrix} \right\}.
    \]

    \item 
    
    \[
    \left\{ 
    \begin{bmatrix}
    1 \\ 
    2 \\ 
    3
    \end{bmatrix}
    ,
    \begin{bmatrix}
    4 \\ 
    5 \\ 
    6
    \end{bmatrix}
    ,
    \begin{bmatrix}
    7 \\ 
    8 \\ 
    9
    \end{bmatrix} \right\}.
    \]

    \item

    \[
    \left\{ 
    \begin{bmatrix}
    1 \\ 
    2 \\ 
    3 \\
    4
    \end{bmatrix}
    ,
    \begin{bmatrix}
    1 \\ 
    1 \\ 
    1 \\
    1
    \end{bmatrix}
    ,
    \begin{bmatrix}
    1 \\ 
    3 \\ 
    1 \\
    2
    \end{bmatrix}
    ,
    \begin{bmatrix}
    2 \\ 
    1 \\ 
    1 \\
    2
    \end{bmatrix} \right\}.
    \]

    \item

    \[
    \left\{ 
    \begin{bmatrix}
    1 \\ 
    2 \\ 
    -1 \\
    0
    \end{bmatrix}
    ,
    \begin{bmatrix}
    0 \\ 
    1 \\ 
    2 \\
    1
    \end{bmatrix}
    ,
    \begin{bmatrix}
    2 \\ 
    3 \\ 
    -4 \\
    -1
    \end{bmatrix} \right\}.
    \]

\end{enumerate}

\subsection{Solución}

Para verificar que uno de los conjuntos de vectores son linealmente independientes de la definición de \textbf{Independencia lineal} tomamos los siguientes enunciados:
\begin{itemize}
    \item El conjunto $A = \{ {v}_{1}, \cdots , {v}_{r} \}$ de vectores ${v}_{i} \in \mathbb{R}^n$, es llamado conjunto linealmente independiente si los vectores ${v}_{1}, \cdots , {v}_{r}$ son linealmente independientes.
    \item Sea $A = \{ {v}_{1}, \cdots , {v}_{r} \} \subset \mathbb{R}^n$ y sea $A = [ {v}_{1} \quad {v}_{2} \quad \cdots \quad {v}_{r} ]$ una matriz de orden $nxr$. $A$ es linealmente independiente $\iff N(A)=\{ 0 \}$
\end{itemize}

\subsubsection{Para a)}
Tenemos los siguientes vectores
\begin{itemize}
    \item ${v}_{1} = (1,2,3)$
    \item ${v}_{2} = (1,0,-1)$
    \item ${v}_{3} = (-2,1,1)$ 
\end{itemize}
y el conjunto $A=\{ {v}_{1}, {v}_{2}, {v}_{3} \}$ expresando el conjunto en una matriz quedaría de la siguiente forma

\[
A = [ {v}_{1} \quad {v}_{2} \quad {v}_{3} ] =
\begin{bmatrix}
1 & 1 & -2 \\ 
2 & 0 & 1 \\
3 & -1 & 1
\end{bmatrix}
\Longrightarrow N(A) = \{ \mathbf{0} \} ?
\]
$$
N(A) = \{ \mathbf{x} \in \mathbb{R}^3 / A \mathbf{x} = \mathbf{0} \}
$$
La solución del sistema lineal estaría representado de la siguiente manera:
\[
\begin{bmatrix}
1 & 1 & -2 \\ 
2 & 0 & 1 \\
3 & -1 & 1
\end{bmatrix}
\begin{bmatrix}
{x}_{1} \\ 
{x}_{2} \\
{x}_{3}
\end{bmatrix} =
\begin{bmatrix}
0 \\ 
0 \\
0
\end{bmatrix}
\]
El cual nos da el siguiente sistema de combinaciones lineales
\setcounter{equation}{0}
\begin{eqnarray}
    {x}_{1} + {x}_{2} - 2{x}_{3} &=& 0 \\
    2{x}_{1} + {x}_{3} &=& 0 \\
    3{x}_{1} - {x}_{2} + {x}_{3} &=& 0
\end{eqnarray}
De lo cual en la ecuación (2) se tiene que ${x}_{3} = -2{x}_{1}$, si reemplazamos este valor en la ecuación (1) se tiene
\begin{eqnarray}
    {x}_{1} + {x}_{2} - 2{x}_{3} &=& 0 \nonumber \\
    {x}_{1} + {x}_{2} - 2(-2{x}_{1}) &=& 0 \nonumber \\
    {x}_{1} + {x}_{2} + 4{x}_{1} &=& 0 \nonumber \\
    {x}_{2} &=& -5{x}_{1} \nonumber
\end{eqnarray}
















\section{Problema}

Sea $A$ una matriz $mxn$, y sea $X \subseteq \mathbb{R}^n$ un subconjunto arbitrario de $\mathbb{R}^n$. La imagen de $X$ bajo la transformación de $A$ se define como el conjunto

$$
A(X) = \{ A \mathbf{x} : \mathbf{x} \in X \}
$$

\begin{itemize}
    \item[(a)] Demuestre que $A(X) \subseteq C(A)$

    \item[(b)] Si $X$ es un subespacio de $\mathbb{R}^n$, pruebe que $A(X)$ es un subespacio de $\mathbb{R}^m$.
\end{itemize}

\section{Problema}

Considere los siguientes vectores en $\mathbb{R}^4$:

\[
    \left\{ 
    \begin{bmatrix}
    1 \\ 
    1 \\ 
    1 \\
    1
    \end{bmatrix}
    ,
    \begin{bmatrix}
    0 \\ 
    1 \\ 
    1 \\
    1
    \end{bmatrix}
    ,
    \begin{bmatrix}
    0 \\ 
    0 \\ 
    1 \\
    1
    \end{bmatrix}
    ,
    \begin{bmatrix}
    0 \\ 
    0 \\ 
    0 \\
    1
    \end{bmatrix} \right\}.
    \]

Aplique el procedimiento de Gram-Schmidt a estos vectores para obtener una base ortonormal de $\mathbb{R}^4$.


