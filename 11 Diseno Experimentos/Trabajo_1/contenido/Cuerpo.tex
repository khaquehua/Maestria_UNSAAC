\newpage
\chapter{INDICACIONES}

\begin{itemize}
	\item El trabajo es académico, ser cuidadoso de la forma en como justificas, redacción y coherencia.
	\item Tener cuidado con la ortografía.
	\item Citar y referenciar en APA 7ma edición.
	\item La fecha de entrega es hasta el jueves 11 hasta las 23:59 hrs, no se aceptaran trabajos después. 
\end{itemize}

Revisar 5 artículos científicos en donde se use diseños experimentales, el área es libre.

Responder a las siguientes interrogantes:

\begin{enumerate}
	\item Título de la investigación
	\item Link de la investigación
	\item Objetivo de la investigación
	\item Material experimental
	\item Unidad experimental
	\item ¿En el desarrollo de la investigación, cumple con los principios básicos de un diseño de experimento? (Indique cuales)
	\item ¿La unidad experimental y unidad de análisis coinciden? Justifique
	\item Diseño experimental
	\item ¿Existe control local? Justifique
	\item El diseño es balanceado o desbalanceado (justifique)
	\item Utiliza algún método de comparación múltiple (Cual)
	\item ¿La investigación verifica el cumplimiento de supuestos?: (Si es afirmativa tu respuesta, indique cuales)
\end{enumerate}

\section{SOLUCIÓN}

Para realizar la resolución de esta activad se esta usando el editor de texto \LaTeX en el cual se hizo las debidas configuraciones en el preámbulo para obtener resultados en formato APA, se presentarán los 5 artículos encontrados, con su respectivo titulo del artículo, resumen y respuesta a las preguntas planteadas.

\subsection{Block design allowed for control of the Hawthorne effect in a randomized controlled trial of test ordering}

El estudio presentado por \cite{verstappen2004block} evalúa el valor de los diseños de bloques incompletos balanceados en la investigación sobre mejora de la calidad y su capacidad para controlar el efecto Hawthorne. Se aleatorizó a equipos médicos generales en tres grupos y se les realizó una intervención sobre la solicitud de pruebas para dos grupos de problemas clínicos (pruebas A y pruebas B). En los dos ensayos con diseño de bloques, se intentó controlar el efecto Hawthorne comparando la intervención completa en ambos grupos en las pruebas A (grupo I) o grupo B (grupo II); los grupos actuaron como controles ciegos entre sí. En el ensayo clásico, se comparó la intervención completa en las pruebas B (grupo II) con un grupo control sin intervención en las pruebas B (grupo III). Los ensayos con diseño de bloques mostraron cambios estadísticamente significativos en el número de pruebas A solicitadas ($p = 0.013$), pero no en el número de pruebas B solicitadas ($p = 0.29$). En el diseño clásico, la intervención completa alcanzó un cambio marginalmente significativo en las pruebas B ($p = 0.068$). El efecto Hawthorne fue el mismo en ambos grupos del diseño de bloques. En el diseño clásico, el efecto podría atribuirse en cierta medida al efecto Hawthorne. El diseño de bloques permitió controlar dicho efecto. Concluyendo que el uso adecuado de diseño de bloques puede ampliar el conocimiento sobre los efectos no específicos en la investigación sobre mejora de la calidad.

En lo cual la resolución de las preguntas planteadas sería:

\begin{enumerate}
	\item \textbf{Título de la investigación:} Block design allowed for control of the Hawthorne effect in a randomized controlled trial of test ordering
	\item \textbf{Link de la investigación:} \url{https://www.jclinepi.com/article/S0895-4356(04)00137-4/fulltext}
	\item \textbf{Objetivo de la investigación:} Evaluar el valor de los diseños de bloques incompletos balanceados para controlar el efecto Hawthorne
	\item \textbf{Material experimental:} Region de un centro de diagnóstico
	\item \textbf{Unidad experimental:} Médicos de cabecera
	\item \textbf{¿En el desarrollo de la investigación, cumple con los principios básicos de un diseño de experimento? (Indique cuales):} Si, con respecto a la repetíción se tuvo a médicos que realizaron las pruebas (PRUEBA A que consiste en pruebas para problemas cardiovasculares, abdominales superiores e inferiores), B para problemas pulmonares y CONTROL), aleatorizados ya que no supieron a que prueba correspondía y se les dieron al azar y control del error ya que se uso el análisis de covarianza.
	\item \textbf{¿La unidad experimental y unidad de análisis coinciden? Justifique:} Si, ya que los grupos de médicos de cabecera vieron si había diferencias en las intervenciones realizadas con respecto al efecto Hwathorne.
	\item \textbf{Diseño experimental:} Diseño bloque incompleto al azar
	\item \textbf{¿Existe control local? Justifique:} Si existe el grupo control, que consiste en aquel que tuvo una intervención mínima.
	\item \textbf{El diseño es balanceado o desbalanceado (justifique):} Desbalanceado, ya que las asignaciones fueron desiguales en los diferentes grupos.
	\item \textbf{Utiliza algún método de comparación múltiple (Cual):} Analisis de covarianza para evaluar el poder y efecto.
	\item \textbf{¿La investigación verifica el cumplimiento de supuestos?: (Si es afirmativa tu respuesta, indique cuales):} No hubo un apartado dedicado al cumplimiento de supuestos, en si se limitó a la presentación de resultados.
\end{enumerate}

