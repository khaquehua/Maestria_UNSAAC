\begin{introduccion}
La gestión de la calidad en salud del (MINSA) indica que la calidad es cuando el proveedor pone énfasis con la calidad técnica de la atención clínica, el usuario hacia la relación interpersonal, comodidades, limpieza, tiempos de espera, disponibilidad del médico, medicinas. Y el administrador prioriza la eficiencia, eficacia y efectividad.

La Guía Metodológica para la Capacitación de los Procesos del Sistema de Suministro de Medicamentos e Insumos en el Ministerio de Salud (DIGEMID) indica que los problemas de stock generalmente se traducen en periodos de desabastecimiento y sobrestock de medicamentos e insumos, lo que afecta la disponibilidad y accesibilidad.

Existen programas informáticos que ayudan a determinar cuándo es el momento para realizar un requerimiento de un insumo o medicamentos basada en registros de información actualizada, precisa y oportuna. Los registros de consumo y stock son importantes para la planificación y programación del abastecimiento de medicamentos e insumos con la finalidad de mantener un nivel óptimo de existencias para garantizar la disponibilidad de estos y evitar la probabilidad de pérdidas debido al deterioro, vencimiento y sustracciones.

Por lo que se realizó un estudio de investigación, aplicado a la gestión de almacén del centro de salud integral ``La Fuente'' en el año 2024, tomando las variables más importantes como lo son la demanda, espacio, pedidos, costo de preparación, costo de almacenamiento, costo de compra y tiempo de entrega de cada producto como variables independientes hacia la variable dependiente de cuándo pedir, cuánto pedir, establecer un punto de reorden y obtener los costos óptimos, para de esta forma implantar una política de inventarios óptima.

Los conceptos y herramientas matemáticas utilizadas abarcan desde cálculo al momento de evaluar la continuidad y encontrar los máximos y mínimos óptimos, estadística como variable aleatoria o distribuciones de probabilidad e investigación operativa como los modelos de inventarios con el propósito de encontrar la política de inventarios óptima que debe tener el almacén del centro de salud integral.

El contenido de este trabajo de investigación se subdivide en cinco capítulos que se describen a continuación:

En el primer capítulo se presenta el planteamiento del problema, la formulación del problema, justificación, los objetivos, las delimitaciones, y la limitación.

En el segundo capítulo, se presenta el marco teórico conceptual que contiene los antecedentes encontrados referentes al trabajo de investigación de forma local, nacional e internacional, seguido de las báses teóricas como son conceptos de cálculos multivariado, variable aleatoria, distribución normal, clasificación de actividades basadas en costos y modelos de inventarios; seguido del marco conceptual y la información general del centro de salud integral.

El tercer capítulo, hace referencia a las hipótesis, la identificación de variables y su operacionalización.

El cuarto capítulo, muestra la metodología implementada para realizar el estudio iniciando desde el tipo, enfoque, alcance, diseño de la investigación, la población de estudio, las técnicas e instrumentos empleados.

El quinto capítulo, muestra los resultados obtenidos desde la parte descriptiva, aplicación del modelo, la política óptima de inventarios y el aplicativo web.

Finalmente se muestran las discusiones, conclusiones, recomendaciones, bibliografía y anexos que complementan el trabajo de investigación.

\end{introduccion}
