%:::::::::::::::::::::::::::::::::::::::::
%CAPITULO
%::::::::::::::::::::::::::::::::::::::::
\titleformat{\chapter}[display]
  {\normalfont\bfseries\centering\large} % Formato del título
  {\MakeUppercase{\chaptertitlename}\ \Roman{chapter}} % Etiqueta (CAPÍTULO I)
  {0pt} % Espacio entre la etiqueta y el título
  {\MakeUppercase} % Eliminamos el \vspace aquí

% Ajustar el espaciado alrededor del título del capítulo
\titlespacing*{\chapter}{0pt}{0pt}{10pt} % Ajustamos los espacios antes y después

% Definir un nuevo tipo de capítulo no numerado centrado
\titleformat{name=\chapter,numberless}[display]
  {\normalfont\bfseries\large\centering} % Formato de fuente y centrado
  {} % No se muestra el número
  {0pt} % Espacio entre la etiqueta y el título
  {\MakeUppercase} % Formato del título

%::::::::::::::::::::::::::::::::::::::::::::::
%:::::::::::::: Modificacion del section y subsection:::
%::::::::::::::::::::::::::::::::::::::::::::::
\titleformat{\section}
  {\normalfont\large\bfseries} % Tamaño large y negrita
  {\thesection} % Número de la sección
  {1em} % Espacio entre el número y el título
  {} % Código antes del título

% Formato para subsection (tamaño large)
\titleformat{\subsection}
  {\normalfont\large\bfseries\itshape} % Tamaño large, negrita e itálica
  {\bfseries\itshape\thesubsection} % Numeración en negrita e itálica
  {1em} % Espacio entre el número y el título
  {} % Código antes del título

% Formato para la subseccion como subseccion
\newcommand{\manualsubsubsection}[2]{%
  \subsubsection*{\normalfont\large\bfseries\itshape #1\ #2}%
  \addcontentsline{toc}{subsubsection}{#1\ #2}%
}


%::::::::::::::::::::::::::::::::::::::::::::::
%CAMBIAR EL NOMBRE DE ENCABEZADOS
%::::::::::::::::::::::::::::::::::::::::::::::
\addto\captionsspanish{\renewcommand{\bibname}{\large{BIBLIOGRAFÍA}}}
\addto\captionsspanish{\renewcommand{\contentsname}{\begin{center} \large ÍNDICE \end{center}}}
\addto\captionsspanish{\renewcommand{\listtablename}{\begin{center} \large ÍNDICE DE TABLAS \end{center}}}
\addto\captionsspanish{\renewcommand{\listfigurename}{\begin{center} \large ÍNDICE DE FIGURAS \end{center}}}

%::::::::::::::::::::::::::::::::::::::::
%Cambiar el salto de pagina en capitulos
%:::::::::::::::::::::::::::::::::::::::
\makeatletter
\patchcmd{\chapter}{\if@openright\cleardoublepage\else\clearpage\fi}{}{}{}
\makeatother

%::::::::::::::::::::::::::::::::::::::::
%Cambiar el formato de tablas y figuras
%:::::::::::::::::::::::::::::::::::::::

%::::::::::::::::::::::::::::::::::::::::
%Segunda opcion termina con dos puntos
%:::::::::::::::::::::::::::::::::::::::

%\DeclareCaptionFormat{custom}
%{%
%    \textbf{\textsl{#1#2}}\\\textsl{ #3}
%}
%\captionsetup{format=custom,
%justification=raggedright,
%  singlelinecheck=false}

%\captionsetup[table]{name={Tabla}}


\captionsetup[table]{name={Tabla},labelfont={bf,sl},textfont={sl},labelsep=newline,justification=raggedright,
singlelinecheck=false}
\captionsetup[figure]{labelfont={bf,sl},textfont={sl},labelsep=newline,justification=raggedright,
singlelinecheck=false}

\counterwithout{figure}{chapter}
\counterwithout{table}{chapter}
\renewcommand{\thetable}{\arabic{table}}
\renewcommand{\thefigure}{\arabic{figure}}

% Ajustar el ancho de la numeración en el índice de tablas y figuras
\setlength{\cfttabnumwidth}{4.5em}  % Ajusta este valor si es necesario
\setlength{\cftfignumwidth}{4.5em}  

% Ajustar el ancho de la separación entre número y texto
\setlength{\cfttabindent}{0em} % Indentación de la numeración
\setlength{\cftfigindent}{0em} 

% Evitar saltos de línea en los títulos largos
\renewcommand{\cfttabpresnum}{\textbf{Tabla~}} 
\renewcommand{\cftfigpresnum}{\textbf{Figura~}}
\renewcommand{\cfttabaftersnum}{\hspace{1em}} % Ajustar espacio después del número
\renewcommand{\cftfigaftersnum}{\hspace{1em}}

% Asegurar que las entradas largas no se dividan en varias líneas
\renewcommand{\cfttabfont}{\normalfont} 
\renewcommand{\cfttabpagefont}{\normalfont}
\fancyhf{}
\fancyfoot[C]{\thepage}
