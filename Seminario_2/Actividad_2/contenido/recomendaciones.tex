\begin{recomendaciones}
	
\begin{enumerate}
\item Al centro de salud se recomienda tener el registro sobre el uso de los productos ya sea utilizando algún software o control para tener un valor más preciso al momento de implantar las políticas de inventarios.
\item Evaluar el impacto de la aplicación de modelos de inventarios para la gestión de inventarios con el fin de medir la importancia de la aplicación de modelos de investigación operativa en centros de salud.
\item De la misma forma recomendar la aplicación de sistemas de inventarios en el área de farmacia del centro de salud integral La Fuente para que también se puedan tomar y optimizar los medicamentos gestionados por farmacia.
\item Asimismo se recomienda a las instituciones de diferentes rubros realizar la aplicación de modelos de inventarios y otras técnicas de investigación operativa para llevar una administración científica y mejorar sus procesos. A través de la generación de aplicativos o el uso del aplicativo generado en RShiny.
\item A futuro realizar la aplicación del modelo de cantidad económica de pedido tomando en cuenta la limitación de almacén, asimismo usar los otros tipos de modelos determinísticos y probabilísticos de inventarios a otras problemáticas de aplicación.
\item Asimismo fomentar por parte de la escuela profesional de matemática y estadística al uso de lenguajes de programación como R, especificamente a la generación de aplicativos web con RShiny a los estudiantes para poder realizar prototipos computacionales con las aplicaciones matemáticas, estadísticas y de investigación operativa. De tal forma que puedan aplicar sus ideas de investigación hacia problemas reales de forma práctica e innovadora.
\end{enumerate}
	
\end{recomendaciones}
