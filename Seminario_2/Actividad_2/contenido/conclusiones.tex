\begin{conclusiones}
	
Según los resultados obtenidos mediante el trabajo de investigación se obtienen las siguientes conclusiones:

\begin{enumerate}
\item Se implementó el modelo clásico de cantidad económica de pedido (EOQ) para optimizar los inventarios de almacén del centro de salud integral La Fuente del Cusco durante el año 2024.
\item Los productos más demandados por el centro de salud integral La Fuente del Cusco en el año 2024 utilizando la metodologías de actividades basadas en costos, son insumos utilizados por el área de oftalmología ocupando el $79.31\%$ de los costos totales de todo el almacén en los que se incluyen los 8 productos clasificados en el grupo A y 10 productos del grupo B, de los cuales las especialidades que tuvieron mayores costos fueron: INLASER ($39.27\%$), FACO ($25.49\%$), insumos generales ($5.97\%$), tintas ($3.78\%$), CROSSLINKING ($2.58\%$), Catarata ($1.52\%$) y esterilización ($0.71\%$). 
\item Los productos seleccionados mediante la metodología (ABC) indicaron un coeficiente de variabilidad menor a 0,20 por lo que se tuvo una demanda determinística y el modelo de inventarios a utilizar en todos los casos era el modelo clásico de cantidad económica de pedido (EOQ).
\item Usando la información brindada por el área de logística y administración del centro de salud integral La Fuente se estableció la cantidad, tiempo y costo óptimo de inventarios asi como el momento de reorden. Pudiendo establecer así la política de inventarios óptima del centro de salud integral.
\item El aplicativo de Shiny apoyará al momento de realizar el monitoreo y seguimiento de los productos, no solo de los productos más demandados, sino también de productos adicionales que se puedan ingresar y pueda apoyar a gestionar una política de inventarios óptima, de la misma forma se incluye el modelo EOQ con escasez y modelo EOQ probabilizado de cantidad pedido.
\item El centro de salud integral La Fuente del Cusco mostró un gran desarrollo con respecto al área administrativa y logística. En el que se espera que la implementación de los modelos de inventarios apoyen en la gestión de las decisiones tomadas acerca de los productos de almacén con base a una administración científica.

\end{enumerate}
	
\end{conclusiones}
