\begin{resumen}
\justifying
Las instituciones sanitarias enfrentan desafíos permanentes en la gestión de inventarios, del cual es fundamental garantizar la disponibilidad oportuna de insumos para contribuir de manera efectiva al bienestar social. En este estudio se analizó los productos de almacén del centro de salud integral ``La Fuente'' utilizando los modelos de inventarios, la investigación es de tipo aplicada, enfoque cuantitativo, diseño no experimental y alcance descriptivo.

Se clasificaron los productos mediante las actividades basadas en costos (ABC), después se evaluo su coeficiente de variabilidad donde se tuvo un ($CV < 0.20$) por el que los productos fueron analizados mediante el modelo clásico de cantidad de pedido económico (EOQ) en los cuales se halló la política óptima de inventarios. Por último se generó un aplicativo en RShiny para analizar productos adicionales así como la inclusión del modelo EOQ con escasez y modelo probabilizado EOQ con reservas.\\
\textbf{Palabras clave:}
Actividades basadas en costo (ABC), modelos de inventarios determinísticos, modelos de inventario probabilísticos, gerencia hospitalaria.

\end{resumen}
