\newpage
\chapter{HIPÓTESIS Y VARIABLES}
\section{Hipótesis}

\section{Hipótesis general}
La inferencia bayesiana indica una causalidad más aproximada para la detección de queratocono.

\section{Hipótesis específicas}

\begin{enumerate}
    \item Las variables más influyentes son las evaluadas en estudios previos de factores asociados al queratocono.
    \item Las redes bayesianas representarán la relación causalidad del queratocono entre variables.
    \item Se evaluará la importancia de usar la inferencia bayesiana frente a la inferencia estadística clásica para determinar si una persona tiene o no tiene queratocono.
    \item El prototipo computacional apoyará en las inferencias de riesgo a nivel individual.
\end{enumerate}

\section{Identificación de variables e indicadores}

Las variables independientes viene a ser las probabilidades de generar queratocono en el nivel que se encuentre, asimismo como la clasificación de presencia o ausencia de esta enfermedad, que va a estar dependiendo principalmente de variables clínicas, demográficas y topográficas.

\begin{enumerate}
    \item \textbf{Variables dependientes}
    \begin{itemize}
        \item Probabilidad de generar queratocono.
        \item Clasificación de presencia o ausencia de queratocono. 
    \end{itemize}
    \item \textbf{Variables independientes} 
    \begin{itemize}
        \item Variables clínicas
        \item Variables demográficas
        \item Variables topográficas 
    \end{itemize}
\end{enumerate}

\begin{landscape} % Iniciar la página en formato horizontal
\subsection{Operacionalización de variables}

\begin{longtable}{p{4cm}p{6.5cm}p{5cm}p{3cm}p{2.5cm}}
    \caption{Matriz de operacionalización de variables} 
    \label{tab:matriz_operacionalizacion} \\

    \hline
    \textbf{Variable} & \textbf{Definición} & \textbf{Indicador} & \textbf{Tipo} & \textbf{Escala} \\
    \hline
    \endfirsthead

    \hline
    \textbf{Variable} & \textbf{Definición} & \textbf{Indicador} & \textbf{Tipo} & \textbf{Escala} \\
    \hline
    \multicolumn{5}{l}{\textbf{Independiente}} \\
    \hline
    \endhead

    \hline
    \endfoot

    \hline
    \endlastfoot

    \multicolumn{5}{l}{\textbf{Dependiente}} \\
    \hline
    Probabilidad de generar queratocono & Posibilidad de generar queratocono & Probabilidad queratocono & Continua & Razón \\
    \hline
    Presencia o ausencia de queratocono & Clasificación de la presencia o ausencia del queratocono & Presencia o ausencia & Dicotómica & Nominal \\
    \hline
    \multicolumn{5}{l}{\textbf{Independiente}} \\
    \hline
    \multirow{4}{*}{Variables clínicas} & \multirow{4}{6.5cm}{Variables tomadas en la revisión de la historia clínica usando test o cuestionarios.} & Ojo seco & Dicotómica & Nominal \\
    \cline{3-5}
    & & Alergias & Dicotómica & Nominal \\
    \cline{3-5}
    & & Frotamiento de ojos & Dicotómica & Nominal \\
    \cline{3-5}
    & & Antecedentes familiares con queratocono & Dicotómica & Nominal \\
    \hline \\
    \multirow{4}{*}{Variables demográficas} & \multirow{4}{6.5cm}{Información general de los pacientes} & Género & Dicotómica & Nominal \\
    \cline{3-5}
    & & Edad & Continua & Razón \\
    \cline{3-5}
    & & Altitud vive & Continua & Razón \\
    \cline{3-5}
    & & Clasificación altitud & Politómica & Ordinal \\
    \hline
    \hline \\
    \multirow{4}{*}{Variables topográficas} & \multirow{4}{6.5cm}{Son indicadores claves que se obtienen en los equipos de oftalmología de forma general} & Esfera topográfica & Continua & Intervalar \\
    \cline{3-5}
    & & Cilindro topográfico & Continua & Razón \\
    \cline{3-5}
    & & Eje topográfico & Continua & Razón \\
    \cline{3-5}
    & & Esfera manifiesta & Continua & Intervalar \\
    \cline{3-5}
    & & Cilindro manifiesto & Continua & Razón \\
    \cline{3-5}
    & & Eje manifiesto & Continua & Razón \\
    \cline{3-5}
    & & Emetropía & Politómica & Nominal \\
    \cline{3-5}
    & & Queratometria media & Continua & Razón \\
    \cline{3-5}
    & & Paquimetría & Continua & Razón \\
    \cline{3-5}
    & & Queratometria media & Continua & Razón \\
    \cline{3-5}
    & & BAD-D & Continua & Razón \\
    \cline{3-5}
    & & RMS & Continua & Razón \\
    \cline{3-5}
    & & IPP media & Continua & Razón \\
    \cline{3-5}
    & & Elevación frontal & Continua & Razón \\
    \cline{3-5}
    & & Elevación posterior & Continua & Razón \\
    \hline
\end{longtable}

\end{landscape}




















