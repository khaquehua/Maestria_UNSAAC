\begin{matriz_consistencia}

\begin{landscape}
\textbf{\Large{ANEXO}}%\\
%\textbf{\normalsize{Matriz de consistencia}}\\
\begin{table}[h!]
\centering
\caption{\textbf{\textsl{Matriz de Consistencia}}}
\footnotesize % Reduce el tamaño de la fuente para que quepa mejor
\begin{tabular}{|p{4.2cm}|p{4.2cm}|p{4.2cm}|p{4.2cm}|p{4.2cm}|}
\hline
\multicolumn{1}{|c|}{\textbf{Problemas}}  & \multicolumn{1}{c|}{\textbf{Objetivos}}  & \multicolumn{1}{c|}{\textbf{Hipótesis}} & \multicolumn{1}{c|}{\textbf{Variables}}   & \multicolumn{1}{c|}{\textbf{Metodología}} \\ \hline
\multicolumn{3}{|c|}{\textbf{General}}                               & \textbf{Dependientes}  & \multirow{4}{=}{
\begin{minipage}{4.2cm}
\justify
$\circ$ \textbf{Tipo:} Aplicado \\
$\circ$ \textbf{Nivel:} Descriptivo Correlacional - Explicativo \\
$\circ$ \textbf{Diseño:} Retrospectivo longitudinal no experimental \\
$\circ$ \textbf{Población:} Todos los productos de almacén. \\
$\circ$ \textbf{Población de estudio:} Productos clasificados en el grupo A mediante (ABC).\\
$\circ$ \textbf{Técnica de recolección de datos:} Revisión documental y/o datos secundarios. \\
$\circ$ \textbf{Instrumento de recolección de datos:} Boletas, ordenes de compra, KARDEX, cotizaciones. \\
$\circ$ \textbf{Método de análisis de datos:} Recopilación de datos en un archivo \textsl{.xlsx}, procesamiento y análisis de datos en el lenguaje de programación R mediante el entorno de desarrollo integrado RStudio.
\end{minipage}
} \\ \cline{1-4}
\multicolumn{1}{|p{4.2cm}|}{¿Cómo es la gestión de inventarios de almacén sobre los productos más demandados del centro de salud integral La Fuente?} & \multicolumn{1}{p{4.2cm}|}{Analizar la gestión de inventarios del almacén en el centro de salud integral La Fuente mediante los modelos de inventarios.} & La gestión de inventarios de almacén sobre los productos más demandados del centro de salud integral ``La Fuente'' es óptimo. & 
    \vspace{0.2cm}
    $\circ$ ¿Cuánto pedir?\vspace{0.2cm}

    $\circ$ ¿Cuándo pedir?\vspace{0.2cm}
  &                       \\ \cline{1-4}
\multicolumn{3}{|c|}{\textbf{Específicos}}                               & \textbf{Independientes} &  \\ \cline{1-4}
\multicolumn{1}{|p{4.2cm}|}{
    $\circ$ ¿Cuáles son los productos más demandados y/o utilizados en el centro de salud integral La Fuente?\vspace{0.2cm}

    $\circ$ ¿Que modelo de inventarios se adecua a los productos más demandados?\vspace{0.2cm}

    $\circ$ ¿Cuál es la cantidad, periodo y costos de pedido óptima para los productos de productos más demandados del centro de salud integral La Fuente mediante el modelo de inventarios?\vspace{0.2cm}

    $\circ$ ¿Cómo desarrollar un seguimiento a los productos más demandados?} & \multicolumn{1}{p{4.2cm}|}{
    $\circ$ Clasificar los productos más demandados del centro de salud integral La Fuente aplicando el análisis de actividades basadas en costo (ABC).\vspace{0.2cm}

    $\circ$ Seleccionar el mejor modelo de inventarios de los productos más demandados.\vspace{0.2cm}

    $\circ$ Determinar mediante el modelo de inventarios la cantidad, el tiempo y costos de pedido óptimo para los productos más demandados del centro de salud integral La Fuente.\vspace{0.2cm}

    $\circ$ Desarrollar un app web en Shiny para realizar el monitoreo y seguimiento de los productos más demandados del centro de salud integral La Fuente.

    } & \multicolumn{1}{p{4.2cm}|}{
    $\circ$ Los productos más demandados y utilizados en el centro de salud integral ``La Fuente'' vienen a ser aquellos usados en el área de oftalmología.\vspace{0.2cm}

    $\circ$ Los modelos de inventarios determinísticos se adecuan a los productos más demandados..\vspace{0.2cm}

    $\circ$ Las cantidades, tiempo y costos de pedidos se optimizarán aplicando modelos de inventarios determinísticos.\vspace{0.2cm}

    $\circ$ Los seguimientos se desarrollarán en función a la demanda en el tiempo para realizar la adquisición de inventarios nuevas que van ingresar al almacén tomando en cuenta el modelo de inventario más óptimo del producto.

    } & \multicolumn{1}{p{4.2cm}|}{
    \vspace{0.2cm}
    $\circ$ Tiempo de reabastecimiento.\vspace{0.2cm}

    $\circ$ Demanda.\vspace{0.2cm}

    $\circ$ Costos.
    }  & \\ \hline
\end{tabular}
\end{table}

\end{landscape}
\end{matriz_consistencia}