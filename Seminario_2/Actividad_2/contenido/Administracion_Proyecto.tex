\newpage
\chapter{ADMINISTRACIÓN DEL PROYECTO}
\section{Asignación de recursos}
\subsection{Recursos humanos}
Responsables de tesis:

\begin{itemize}
	\item Kevin Heberth Haquehua Apaza 
\end{itemize}

\subsection{Recursos materiales}
\begin{enumerate}[label=\Alph*]
	\item \textbf{Equipos:}
	\begin{itemize}
		\item Laptop
		\item Impresora
		\item USB 
	\end{itemize}
	\item \textbf{Materiales:}
	\begin{itemize}
	 	\item Papel bond
	 	\item Lapicero
	 	\item Documentos: Boletas, ordenes de compra, KARDEX, documentos de movimientos entre otros. 
	 \end{itemize} 
\end{enumerate}

\newpage
\subsection{Presupuestos}

\begin{table}[h!]
    \caption{Lista de productos y sus costos}
    \begin{tabular}{|l|c|c|c|}
        \hline
        \textbf{Descripción} & \textbf{Cantidad} & \textbf{Precio Unitario} & \textbf{Costo total} \\
        \hline
        Laptop & 1 & S/.2000.00 & S/.2000.00 \\
        \hline
        Impresora & 1 & S/.750.00 & S/.750.00 \\
        \hline
        USB & 1 & S/.20.00 & S/.20.00 \\
        \hline
        Papel bond (Paquete) & 1 & S/.15.00 & S/.15.00 \\
        \hline
        Lapiceros & 3 & S/.1.00 & S/.3.00 \\
        \hline
        Documentos & 120 & S/.0.20 & S/.24.00 \\
        \hline
        \textbf{TOTAL} & - & - & \textbf{S/.2812.00} \\
        \hline
    \end{tabular}
    
\end{table}

\subsection{Cronograma de actividades}

\begin{itemize}
	\item  Elaboración y aprobación de proyecto de tesis
	\item Analizar y procesar información
	\item Redactar el informe final
	\item Presentación
	\item Sustentación
\end{itemize}

\begin{table}[h!]
    \caption{Plan de actividades}
    \begin{tabular}{|>{\raggedright\arraybackslash}m{4cm}|>{\centering\arraybackslash}m{2cm}|>{\centering\arraybackslash}m{2cm}|>{\centering\arraybackslash}m{2cm}|>{\centering\arraybackslash}m{2cm}|}
        \hline
        \textbf{Actividades} & \textbf{Octubre} & \textbf{Noviembre} & \textbf{Diciembre} & \textbf{Enero} \\
        \hline
        \makecell[tl]{Elaboración y \\ aprobación del \\ proyecto de tesis} & \checkmark & & & \\
        \hline
        \makecell[tl]{Análisis y proceso \\ de la información} & & \checkmark & & \\
        \hline
        \makecell[tl]{Redacción del \\ informe final} & & & \checkmark & \\
        \hline
        Presentación & & & \checkmark & \\
        \hline
        Sustentación & & & & \checkmark \\
        \hline
    \end{tabular}
\end{table}


%\newpage
\begin{landscape} % Iniciar la página en formato horizontal
\subsection{Matriz de consistencia}


\begin{table}[h!]
    %\centering
    \caption{\textbf{\textsl{MODELOS DE INVENTARIOS APLICADO AL CONTROL DE ALMACEN DEL CENTRO DE SALUD INTEGRAL LA FUENTE, 2023 - CUSCO - PERÚ}}}
    \begin{tabular}{|>{\raggedright\arraybackslash}m{6.3cm}|>{\raggedright\arraybackslash}m{6.3cm}|>{\raggedright\arraybackslash}m{6.3cm}|>{\raggedright\arraybackslash}m{2.9cm}|}
        \hline
        \textbf{PROBLEMA} & \textbf{OBJETIVOS} & \textbf{HIPOTESIS} & \textbf{VARIABLE} \\
        \hline

        {\footnotesize ¿Cómo es la gestión de inventarios de almacén sobre los productos más demandados del centro de salud integral ``La Fuente''?} &

        {\footnotesize Analizar la gestión de inventarios de almacén de los productos de almacén del centro de salud integral ``La Fuente'' mediante modelos de inventarios.} & 

        {\footnotesize La gestión de inventarios de almacén del centro de salud integral ``La Fuente'' es óptimo.} & 
        {\footnotesize \textbf{Dependientes:}}\vspace{0.3cm}

        {\footnotesize Costo total por unidad de tiempo $(CTU(y))$} \\
        \hline

            {\footnotesize $\circ$ ¿Cuáles son los productos más demandados y/o utilizados en el centro de salud integral La Fuente?}\vspace{0.3cm}

            {\footnotesize $\circ$ ¿Cómo es la gestión de los productos más demandados del centro de salud integral La Fuente?}\vspace{0.3cm}

            {\footnotesize $\circ$ ¿Cuál es la cantidad y periodo de pedido óptima para los insumos de productos más demandados del centro de salud integral La Fuente?}\vspace{0.3cm}

            {\footnotesize $\circ$ ¿Cómo son los costos de los productos más demandados del centro de salud integral ``La Fuente'' mediante el modelo de inventarios?} & 

        {\footnotesize $\circ$ Clasificar los productos más demandados del centro de salud integral La Fuente aplicando el principio de Pareto o análisis ABC.}\vspace{0.3cm}

            {\footnotesize $\circ$ Describir mediante modelos de inventarios los insumos más demandados del centro de salud integral ``La Fuente''}\vspace{0.3cm}

            {\footnotesize $\circ$ Determinar mediante el modelo de inventarios la cantidad y el tiempo de pedido óptimo para los insumos más demandados del centro de salud integral ``La Fuente''}\vspace{0.3cm}

            {\footnotesize $\circ$ Optimizar los costos de los productos más demandados del centro de salud integral ``La Fuente'' mediante el modelo de inventarios.} & 

        {\footnotesize $\circ$ Los productos más demandados vienen a ser aquellos del área de oftalmología y cirugías.}\vspace{0.3cm}

            {\footnotesize $\circ$ La gestión de inventarios de los insumos más demandados del centro de salud integral ``La Fuente'' es óptimo.}\vspace{0.3cm}

            {\footnotesize $\circ$ Los tiempos de pedidos y costos de insumos más demandados se reducirán aplicando modelos de inventarios.}\vspace{0.3cm}

            {\footnotesize $\circ$ Los costos de almacén del centro de salud integral ``La Fuente'' se optimizarán aplicando modelos de inventarios.} & 
        {\footnotesize \textbf{Independientes:}}\vspace{0.3cm}

        {\footnotesize $\circ$ Demanda $(D)$}\vspace{0.3cm}

        $\circ$ Costo de compra $(C)$\vspace{0.3cm}

        $\circ$ Costo de preparación $(K)$\vspace{0.3cm}

        $\circ$ Costo de almacenamiento $(h)$} \\
        \hline
    \end{tabular}
\end{table}
\end{landscape} % Finalizar la página horizontal