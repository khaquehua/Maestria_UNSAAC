\begin{presentacion}
\noindent
  \textbf{SEÑOR:} DECANO DE LA FACULTAD DE CIENCIAS QUÍMICAS, FÍSICAS Y MATEMÁTICAS DE LA UNIVERSIDAD NACIONAL DE SAN ANTONIO ABAD DEL CUSCO\\
  \textbf{SEÑOR:} DIRECTOR DE LA ESCUELA PROFESIONAL DE MATEMÁTICA\\
  \textbf{SEÑORES:} DOCENTES MIEMBROS DEL JURADO

\noindent
Con la conformidad del reglamento de grados y títulos de la Escuela Profesional de Matemática, pongo a consideración el presente trabajo de tesis intitulado: \textbf{ ``MODELO DE INVENTARIOS APLICADO AL CONTROL DE ALMACÉN DEL CENTRO DE SALUD INTEGRAL LA FUENTE DEL CUSCO, 2024''} para optar al Título Profesional de \textbf{LICENCIADO EN MATEMÁTICA MENCIÓN ESTADÍSTICA}.

\noindent
El trabajo de investigación tiene por objetivo desarrollar una política de inventarios óptima para la gestión de productos e insumos de almacén mediante las técnicas de investigación operativa usando modelos de inventarios ya que evalúa la demanda y las restricciones que tiene cada producto del inventario. 
Se utilizó fuentes de información secundaria como registros del KARDEX y órdenes de compra del año 2024 del centro de salud integral ``La Fuente''. 
Para el análisis se utilizó el lenguaje de programación R bajo el entorno de desarrollo integrado RStudio.
En la presente investigación se aplican diversas técnicas de estadística e investigación operativa, en la parte exploratoria se utilizará el análisis basadas en costos (ABC) con el diagrama de Pareto en la que se identifiquen los productos con mayor porcentaje de costos del centro de salud para posteriormente aplicar los modelos de inventarios determinísticos y probabilísticos hacia los productos seleccionados mediante la metodología.

\noindent
Haciendo la realización de lo propuesto y verificando la aplicación de técnicas estadísticas, matemáticas y de investigación operativa hacia problemas reales, quedo muy agradecido a los señores miembros del jurado por las observaciones que tengan a formular.


\begin{flushright}
  Atentamente,
  
  Br. Kevin Heberth Haquehua Apaza
\end{flushright}


\end{presentacion}
