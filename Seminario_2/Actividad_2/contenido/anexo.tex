% Capítulos personalizados
\makeatletter
\renewcommand\chapter[1]{%
    \clearpage
    \thispagestyle{empty}
    \phantomsection
    \refstepcounter{chapter}% Actualiza y vincula
    \addcontentsline{toc}{chapter}{#1}%
    \vspace*{\fill}
    \begin{center}
        \normalfont\LARGE\bfseries #1
    \end{center}
    \vspace{\fill}
    \clearpage
    \setcounter{section}{0}
}
\makeatother

% Numeración de secciones en letras
\renewcommand\thesection{\alph{section}.}

% Secciones personalizadas
\makeatletter
\renewcommand\section[1]{%
    \phantomsection
    \refstepcounter{section}%
    \addcontentsline{toc}{section}{\protect\numberline{\thesection}#1}%
    {\normalfont\Large\bfseries \thesection \hspace{0.5cm} #1}%
}
\makeatother

\chapter{ANEXOS}

\begin{landscape}
\section{Matriz de consistencia}
%\addcontentsline{toc}{chapter}{ANEXOS}
%\textbf{\Large{ANEXOS}}\\
%\addcontentsline{toc}{section}{A. Matriz de consistencia}
%\textbf{A. Matriz de consistencia}
\begin{table}[h!]
\centering
%\caption{Matriz de Consistencia}
\footnotesize % Reduce el tamaño de la fuente para que quepa mejor
\begin{tabular}{|p{4.5cm}|p{4.5cm}|p{4.5cm}|p{3cm}|p{4.2cm}|}
\hline
\multicolumn{1}{|c|}{\textbf{Problemas}}  & \multicolumn{1}{c|}{\textbf{Objetivos}}  & \multicolumn{1}{c|}{\textbf{Hipótesis}} & \multicolumn{1}{c|}{\textbf{Variables}}   & \multicolumn{1}{c|}{\textbf{Metodología}} \\ \hline
\multicolumn{3}{|c|}{\textbf{General}}                               & \textbf{Dependientes}  & \multirow{4}{=}{
\begin{minipage}{4.2cm}
\justify
$\circ$ \textbf{Ámbito de estudio:} La Fuente Centro de Salud Integral - Cusco - Perú \\
$\circ$ \textbf{Tipo y nivel:} aplicada no experimental retrospectiva, relacional-causal\\
$\circ$ \textbf{Unidad de análisis:} Pacientes atendidos en el centro de salud integral\\
$\circ$ \textbf{Población:} Pacientes atendidos que cumplan los criterios de inclusión\\
$\circ$ \textbf{Muestreo:} Aleatorio simple\\
$\circ$ \textbf{Instrumento recolección de datos:}Historia clínica y resultados topográfico de PENTACAM\\
$\circ$ \textbf{Análisis e interpretación de datos:}Archivo de extensión \textsl{.xlsx} y RStudio.
\end{minipage}
} \\ \cline{1-4}
\multicolumn{1}{|p{4.5cm}|}{¿Cómo comparar la inferencia estadística clásica con la inferencia bayesiana?} & \multicolumn{1}{p{4.5cm}|}{Comparar la inferencia estadística clásica con la inferencia bayesiana para la detección del queratocono} & La inferencia bayesiana indica una causalidad más aproximada para la detección de queratocono. & 
    \vspace{0.2cm}
    $\circ$ Probabilidad de presencia de queratocono\vspace{0.2cm}

    $\circ$ Clasificación de presencia o ausencia de queratocono\vspace{0.2cm}
  &                       \\ \cline{1-4}
\multicolumn{3}{|c|}{\textbf{Específicos}}                               & \textbf{Independientes} &  \\ \cline{1-4}
\multicolumn{1}{|p{4.5cm}|}{
    $\circ$ ¿Qué variables clínicas, demográficas y topográficas se asocian en la detección del queratocono?\vspace{0.2cm}

    $\circ$ ¿Cómo se obtiene la inferencia de probabilidad en la presencia o ausencia de queratocono a través de las diversas variables?\vspace{0.2cm}

    $\circ$ ¿Como afecta la decisión de usar la inferencia estadística clásica o bayesiana en la detección de queratocono?\vspace{0.2cm}

    $\circ$ ¿Cómo implementar las inferencias de riesgo a nivel individual?} & \multicolumn{1}{p{4.5cm}|}{
    $\circ$ Determinar las variables clínicas, demográficas y topográficas asociadas que apoyen en la detección del queratocono.\vspace{0.2cm}

    $\circ$ Construir una red bayesiana estructurada que infiera la probabilidad de presencia o ausencia del queratocono a partir de las variables observadas y como estas se actualizan al variar sus parámetros.\vspace{0.2cm}

    $\circ$ Evaluar la detección de queratocono para futuros pacientes utilizando la inferencia estadística clásica y la inferencia bayesiana.\vspace{0.2cm}

    $\circ$ Implementar un prototipo computacional que visualize las inferencias de riesgo de queratocono a nivel individual.

    } & \multicolumn{1}{p{4.5cm}|}{
    $\circ$ Las variables más influyentes son las evaluadas en estudios previos de factores asociados al queratocono.\vspace{0.2cm}

    $\circ$ Las redes bayesianas representarán la relación causalidad del queratocono entre variables.\vspace{0.2cm}

    $\circ$ Se evaluará la importancia de usar la inferencia bayesiana frente a la inferencia estadística clásica para determinar si una persona tiene o no tiene queratocono.\vspace{0.2cm}

    $\circ$ El prototipo computacional apoyará en las inferencias de riesgo a nivel individual.

    } & \multicolumn{1}{p{3cm}|}{
    \vspace{0.2cm}
    $\circ$ Variables demográficas.\vspace{0.2cm}

    $\circ$ Variables clínicas.\vspace{0.2cm}

    $\circ$ Variables topográficas.
    }  & \\ \hline
\end{tabular}
\end{table}
\end{landscape}

