\newpage
\chapter{PLANTEAMIENTO DEL PROBLEMA}

\section{Descripción de la situación problemática}
Uno de los problemas más frecuentes que enfrentan las instituciones en general es la gestión de inventarios \citep{Flores_Tapia_2023}. Factores asociados son la información limitada o porque no se aplicaron técnicas de investigación de operaciones favorables en la toma de decisiones. Asi mismo se ven obligadas a cuidar su inversión y realizar operaciones adecuadas a través del cuidado de sus inventarios. \citep{Yanque_Vara_2023}

\cite{magalon2016gerencia} indican que la complejidad del hospital como organización ha aumentado sin cesar, en la medida que el conocimiento médico logra nuevos avances que requieren tecnología compleja y costosa. Adicionalmente los sistemas de salud se hacen más exigentes y competitivos, con el fin de cumplir los objetivos propuestos y satisfacer las necesidades de las comunidades. Se deben establecer las razones financieras de posición, endeudamiento, eficiencia operativa y rotación de inventarios, para llegar a las razones de rentabilidad, y así hacer análisis de tendencia financiera para el futuro cercano.

Al ser los productos de almacén una proporción significativa de activos, es necesario evaluar los tiempos entre demandas, abastecimientos y niveles de servicio.

Además de que es importante ver la importancia que tienen los productos de inventario en centros de salud como por ejemplo \cite{prawda2000metodos} toma el caso de una emergencia en un hospital que requiera 10 litros de sangre ARh positiva y, por no tenerla en inventario en ese preciso momento, se muere el enfermo. Los costos en la administración serían en algunos casos leves y en otros más extremos fuertes como en el cual tengan que llegar a pagar una indemnización a los deudos repercutiendo en la entidad encargada.

En Perú el encargado de la dirección de almacén y distribución es el CENARES ``Centro Nacional de Abastecimiento de Recursos Estratégicos en Salud'' perteneciente al Ministerio de Salud encargada de la recepción, almacenamiento, distribución de los recursos estratégicos en salud y otros bienes adquiridos, garantizando su custodia, calidad y distribución oportuna a los establecimientos del Ministerio de Salud (MINSA), el Seguro Social de Salud (EsSalud), los organismos dependientes de los gobiernos regionales y de otras entidades de salud.

En el centro de salud integral La Fuente el ente encargado de las principales direcciones de almacén es el área de logística. En el cual en los últimos años gestionaron los productos de almacén mediante los registros, cotizaciones, seguimientos entre otros para las distintas áreas del centro de salud (oftalmología y odontología principalmente), teniendo resultados muy favorables para el centro de salud ya que se pudo tener un mejor control de la gestión de inventarios.

Ahora se desea evaluar si los productos registrados y obtenidos por los distintos proveedores son óptimos de tal forma que se tenga un balance entre el abastecimiento de productos con sus precios así como la demanda de los pacientes al momento de necesitar los productos en las diversas especialidades. Los modelos de inventarios ayudan a reducir o minimizar los niveles de inventario requeridos en las empresas.

Por tal motivo se presenta la siguiente investigación en el que se pretende mejorar el abastecimiento de almacén así como la evaluación de los productos obtenidos, del mismo modo saber el momento adecuado para realizar los pedidos de almacén de productos.

\section{Formulación del problema}
\subsection{Problema general}
¿Cómo implementar los modelos de inventarios para optimizar el control de almacén del centro de salud integral La Fuente del Cusco durante el año 2024?

\subsection{Problemas específicos}
El estudio toma en cuenta los siguientes problemas específicos:

\begin{itemize}
	\item ¿Cuáles son los productos más demandados y/o costosos en el centro de salud integral La Fuente del Cusco?
	\item ¿Qué modelo de inventarios se adecua a los productos más demandados y/o costosos del centro de salud integral La Fuente del Cusco?
	\item ¿Cuál es la cantidad, periodo y costos de pedido óptimo para los productos más demandados y/o costosos del centro de salud integral La Fuente del Cusco mediante el modelo de inventarios?
	\item ¿Cómo desarrollar un aplicativo web que apoye con el monitoreo y seguimiento de los productos del centro de salud integral La Fuente del Cusco?
\end{itemize}

\section{Justificación de la investigación}
El estudio emplea técnicas de investigación operativa, específicamente los modelos de inventarios como herramienta principal de análisis, el cual contribuye en aplicaciones de modelos matemáticos y estadísticos evaluando sus principios, definiciones, y casos de uso de cada modelo. Facilitando la selección del modelo más adecuado según las características y necesidades del sistema analizado.

Usar modelos de inventarios como estrategia ayudan a establecer razones financieras, reduciendo costos por mantenimientos, sobrestock y tiempo que se le pueda dedicar a estas actividades. De tal manera que también se ve un incremento financiero para el centro de salud generando beneficios económicos.

Asimismo en el ámbito de la salud, tener los insumos necesarios en el tiempo solicitado constituye un componente fundamental en los servicios de salud, garantizando la calidad de la atención brindada y asimismo evitando contratiempo que puedan repercutir en consecuencias muy graves tanto en la parte de la salud del paciente como en la parte administrativa que conlleve. Por lo que mantener un nivel óptimo de inventario basada en los costos, la demanda y una gestión eficiente garantiza la disponibilidad continua y minimiza riesgos de desabastecimiento.
\newpage
\section{Objetivos de la investigación}
\subsection{Objetivo general}
Implementar los modelos de inventarios para optimizar el control de almacén del centro de salud integral La Fuente del Cusco durante el año 2024.
\subsection{Objetivos específicos}
\begin{itemize}
	\item Clasificar los productos más demandados y/o costosos del centro de salud integral La Fuente del Cusco aplicando el análisis de actividades basadas en costo (ABC).
	\item Identificar el mejor modelo de inventarios para los productos más demandados y/o costosos del centro de salud integral La Fuente del Cusco.
	\item Determinar la cantidad, el tiempo y costos de pedido óptimo mediante el modelo de inventarios identificado para los productos más demandados y/o costosos del centro de salud integral La Fuente del Cusco.
	\item Desarrollar un aplicativo web en Shiny para realizar el monitoreo y seguimiento de los productos del centro de salud integral La Fuente del Cusco.
\end{itemize}

\section{Delimitación de la investigación}

\subsection{Delimitación espacial}
El presente estudio es está llevando a cabo en el centro de salud integral ``La Fuente'', ubicado geográficamente en el distrito de San Jerónimo, provincia de Cusco y región del Cusco. La unidad de análisis son los inventarios que se encuentran en almacén del centro de salud.
\subsection{Delimitación temporal}
Al ser un estudio transversal se tomará en cuenta la recopilación de información en el área de logística del centro de salud integral en el año 2024. En los cuales se toma información que contenga las entradas, salidas de productos, ventas, métricas de espacio, entre otros.

\section{Limitación de la investigación}

Para el estudio de investigación se presentaron la siguiente limitación:
\begin{itemize}
	\item Escasa literatura de trabajos de investigación sobre modelo de inventarios con enfoque al área de estadística o matemática.
	\item Información no organizada conforme a la metodología de modelos de inventarios en el centro de salud, debido a que no se tuvieron estudios previos realizados.  
\end{itemize}
