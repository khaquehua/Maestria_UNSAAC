\begin{discusiones}
Según el estudio de \cite{gallardo2016propuesta}, titulado \textsl{''Propuesta de mejora para la gestión de inventarios de sociedad repuestos España limitada''} se tuvo un total de 2994 productos estudiados, en los cuales fueron 319 productos clasificados como ``A'' representando el 70$\%$ de las ventas, estos productos fueron analizados de los cuales se tuvieron 102 productos con demanda determinística y 217 productos con demanda probabilística. En la investigación desarrollada se tuvo de un total de 471 productos analizados de los cuales se tuvo 8 productos clasificados como ``A'' representando el $68.55\%$, asimismo se tomaron productos con mayores costos del grupo ``B'' para tener un análisis de 18 productos, en las cuales todos los productos tuvieron una demanda determinística.

La investigación de \cite{loja2015propuesta} titulado \textsl{''Propuesta de un sistema de gestión de inventarios para la empresa FEMARPE CIA.LTDA''} indico que la empresa no lleva fundamentos científicos en las acciones administrativas, sin tener la información necesaria para la aplicación de metodologías propuestas como las 5 S Japonesas, por lo que se recomendó una adecuada organización, estandarización de insumos, autodisciplina del personal, iniciando con la clasificación ABC enfatizando a que productos debe darle la prioridad para evitar gastos adicionales al momento de realizar la aplicación de métodos científicos administrativos. En el estudio realizado en el centro de salud integral La Fuente se observó que se tuvo un gran desarrolló en la parte administrativa, ya que tuvieron un control y registro de todos los movimientos realizados por parte del área de logística, siendo un gran progreso que tuvieron a comparación de los inicios que tuvo la entidad, sin embargo no se tuvo la información exacta en cuanto a la demanda o uso que se daba de cada producto, ya que tuvo que realizarse una estimación en base a los  procedimientos realizados que generalmente usarán los productos, por lo que para este caso se espera que el centro de salud a futuro tenga el registro de estos procesos de productos ya sea usando un software o detallando los registros de salidas por parte del personal que realiza la solicitación.

Por otra parte, al igual que el estudio de \cite{caceres2010propuesta} titulado \textsl{''Propuesta de un modelo de gestión de inventarios que permita mejorar la planeación y la distribución de las medicinas a las farmacias de un hospital''}, es importante tomar en cuenta la composición administrativa y su relación con otras áreas como lo es en el caso de farmacia para poder implantar una mejor política de inventarios, al igual que se tomo la información del KARDEX para ver los movimientos realizados y poder realizar la clasificación por actividades basadas en costos (ABC) en los cuales a pesar de que el centro de salud no tuvo problemas de sobre stock, desabastecimientos y programaciones de abastecimientos al momento de realizar las solicitudes, se espera que la política de inventarios propuesta mejore la gestión de inventarios del centro de salud y asimismo también implantar la metodología para el área de farmacia.

De la misma forma que el estudio de \cite{caballero2007control} titulada \textsl{''Control de inventario para una empresa de capacitación en el área de salud''} se desarrollaron conceptos de sistemas de inventarios para mejorar el desempeño de actividades en la política de inventarios, en el cual el área de logística pueda gestionar eficientemente el sistema de inventarios obteniendo controles óptimos y mejor administración de insumos.

Para este caso no se realizó la aplicación del modelo de inventarios con limitación de almacén como el desarrollado en el trabajo de \cite{koper2017analisis} titulado \textsl{''Análisis del inventario de almacén en la distribuidora Valle Sur S.A. - 2017; mediante el programa de inventario de almacén INVAL''} debido a que a pesar de que los productos tuvieron un área de ocupación en almacén, estos no se ven restringidos obligatoriamente a este espacio, ya que a comparación del estudio en el que se tuvieron que tomar en cuenta los pesos que tuvo cada producto como una limitante al momento de realizar los pedidos, el centro de salud integral no tuvo esos limitantes para los productos, en cambio se recomienda que se vea una mejor opción de almacenamiento especialmente a los productos que tuvieron una clasificación ``A'' mediante las actividades basadas en costos (ABC). Sin embargo a futuro se debería tomar en cuenta estas restricciones de espacio especialmente tomando en cuenta la visión del centro de salud sobre su desarrolló.
	
\end{discusiones}