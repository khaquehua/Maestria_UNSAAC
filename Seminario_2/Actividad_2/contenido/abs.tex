\begin{abs}
\justifying
Healthcare institutions face ongoing challenges in inventory management, where ensuring the timely availability of supplies is essential to effectively contribute to social welfare. This study analysed the warehouse products of the Comprehensive Health Centre “La Fuente” through the application of inventory models. The research was applied in nature, adopted a quantitative approach, employed a non-experimental design, and had a descriptive scope.

Products were classified using Activity-Based Costing (ABC), after which their coefficient of variation was assessed, obtaining a result below 0.20 ($CV < 0.20$). Consequently, the products were analysed using the classical Economic Order Quantity (EOQ) model, from which the optimal inventory policy was determined. Finally, a web application in RShiny was developed to analyse additional products, as well as to incorporate the EOQ model with shortages and the probabilistic EOQ model with reserves. \\
\textbf{Keywords:}
Activity-Based Costing (ABC), deterministic inventory models, probabilistic inventory models, hospital management.

\end{abs}
