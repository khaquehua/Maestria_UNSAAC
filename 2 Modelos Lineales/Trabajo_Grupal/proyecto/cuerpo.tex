
%\tableofcontents
\newpage

%%%%%%%%%%%%%%%%%%%%%%
%%%%%%%%% Capítulo I
%%%%%%%%%%%%%%%%%%%%%%%

\section{Problema}

Sea el vector aleatorio $\mathbf{v} \sim N_4(\boldsymbol{\mu}, \Sigma)$,

$$\boldsymbol{\mu} = 
\begin{pmatrix} 
2 \\ 
5 \\ 
-2 \\ 
1 
\end{pmatrix}, 
\quad 
\Sigma = 
\begin{pmatrix} 
9 & 0 & 3 & 3 \\ 
0 & 1 & -1 & 2 \\ 
3 & -1 & 6 & -3 \\ 
3 & 2 & -3 & 7 
\end{pmatrix}.$$

Si $\mathbf{v}$ es particionada como $\mathbf{v} = (y_1, y_2, x_1, x_2)'$,

$$
\mathbf{\mu_y} = 
\begin{pmatrix} 
2 \\ 
5 
\end{pmatrix}, 
\quad 
\mathbf{\mu_x} = 
\begin{pmatrix} 
-2 \\ 
1 
\end{pmatrix}, 
$$

$$
\Sigma_{yy} = 
\begin{pmatrix} 
9 & 0 \\ 
0 & 1 
\end{pmatrix}, 
\quad 
\Sigma_{yx} = 
\begin{pmatrix} 
3 & 3 \\ 
-1 & 2 
\end{pmatrix}, 
\quad 
\Sigma_{xx} = 
\begin{pmatrix} 
6 & -3 \\ 
-3 & 7 
\end{pmatrix}. 
$$

Obtener

%%%%%%%%% sección 1.1

\subsection{Esperanza}

$\mathbb{E}(\mathbf{y}|\mathbf{x})$

\subsubsection{Solución}

Utilizando el teorema 4:\\
\textbf{Teorema 4:} Si $y$ y $x$ son normales multivariadas conjuntas con $\Sigma_{yx} \neq 0$, entonces la distribución condicional de $y$ dado $x$, $f(y | x)$, es normal multivariada con vector de medias y matriz de covarianzas.

$$\mathbb{E}(\mathbf{y}|\mathbf{x}) = \mu_y + \Sigma_{yx} \Sigma_{xx}^{-1} (\mathbf{x} - \mu_x)$$

\begin{enumerate}
    \item \textbf{Invertir }$\Sigma_{xx} = \Sigma_{xx}^{-1}$, usando la fórmula de inversa de una matriz

    $$
    \Sigma_{xx}^{-1} = \frac{1}{\det(\Sigma_{xx})} 
    \begin{pmatrix} 
    \sigma_{22} & -\sigma_{12} \\ 
    -\sigma_{21} & \sigma_{11} 
    \end{pmatrix}.
    $$

    Calculamos el determinante de $\Sigma_{xx}$

    $$
    \det(\Sigma_{xx}) = (6)(7) - (-3)(-3) = 42 - 9 = 33.
    $$

    De tal forma que tenemos

    $$
    \Sigma_{xx}^{-1} = \frac{1}{33} 
    \begin{pmatrix} 
    7 & 3 \\ 
    3 & 6 
    \end{pmatrix} =
    \begin{pmatrix} 
    \frac{7}{33} & \frac{3}{33} \\ 
    \frac{3}{33} & \frac{6}{33} 
    \end{pmatrix}
    $$

    \item \textbf{Multiplicar }$\Sigma_{yx} \Sigma_{xx}^{-1}$

    $$
    \Sigma_{yx} \Sigma_{xx}^{-1} = 
    \begin{pmatrix} 
    3 & 3 \\ 
    -1 & 2 
    \end{pmatrix}
    \begin{pmatrix} 
    \frac{7}{33} & \frac{3}{33} \\ 
    \frac{3}{33} & \frac{6}{33} 
    \end{pmatrix} =
    \begin{pmatrix} 
    \frac{30}{33} & \frac{27}{33} \\ 
    -\frac{1}{33} & \frac{9}{33} 
    \end{pmatrix}
    $$

    \item \textbf{Calcular }$(\mathbf{x} - \mu_x)$

    $$
    \mathbf{x} - \mu_x = 
    \begin{pmatrix} 
    x_1 \\ 
    x_2 
    \end{pmatrix}
    -
    \begin{pmatrix} 
    -2 \\ 
    1 
    \end{pmatrix}
    =
    \begin{pmatrix} 
    x_1 + 2 \\ 
    x_2 - 1 
    \end{pmatrix}
    $$

    \item \textbf{Calcular } $\Sigma_{yx} \Sigma_{xx}^{-1} (\mathbf{x} - \mu_x)$

    $$
    \Sigma_{yx} \Sigma_{xx}^{-1} (\mathbf{x} - \mu_x) = 
    \begin{pmatrix} 
    \frac{30}{33} & \frac{27}{33} \\ 
    -\frac{1}{33} & \frac{9}{33} 
    \end{pmatrix}
    \begin{pmatrix} 
    x_1 + 2 \\ 
    x_2 - 1 
    \end{pmatrix}
    $$

    $$
    \Sigma_{yx} \Sigma_{xx}^{-1} (\mathbf{x} - \mu_x) = 
    \begin{pmatrix} 
    \frac{30}{33}(x_1+2) + \frac{27}{33}(x_2-1) \\
    (-\frac{1}{33})(x_1+2) + \frac{9}{33}(x_2-1) 
    \end{pmatrix} = 
    \begin{pmatrix} 
    \frac{30}{33} x_1 + \frac{27}{33}x_2 + 1 \\
    -\frac{1}{33}x_1 + \frac{9}{33}x_2 - \frac{1}{3} 
    \end{pmatrix}
    $$

    \item \textbf{Sumar }$\mu_y$

    $$
    \mathbb{E}(\mathbf{y}|\mathbf{x}) = \mu_y + \Sigma_{yx} \Sigma_{xx}^{-1} (\mathbf{x} - \mu_x)
    $$

    $$
    \mathbb{E}(\mathbf{y}|\mathbf{x}) = \begin{pmatrix} 
    2 \\ 
    5 
    \end{pmatrix}
    +
    \begin{pmatrix} 
    \frac{30}{33} x_1 + \frac{27}{33}x_2 + 1 \\
    -\frac{1}{33}x_1 + \frac{9}{33}x_2 - \frac{1}{3} 
    \end{pmatrix} = 
    \begin{pmatrix} 
    \frac{30}{33}x_1 + \frac{27}{33}x_2 + 3 \\ 
    - \frac{1}{33}x_1 + \frac{9}{33}x_2 + \frac{154}{33}
    \end{pmatrix}
    $$

\end{enumerate}

%%%%%%%%% sección 1.2

\subsection{Covarianza}

$\text{cov}(\mathbf{y}|\mathbf{x})$

\subsubsection{Solución}

De la misma forma utilizamos la definición del teorema 4 de covarianza.

$$\text{cov}(\mathbf{y}|\mathbf{x}) =
\Sigma_{yy} - \Sigma_{yx} \Sigma_{xx}^{-1} \Sigma_{xy}$$

\begin{enumerate}
    \item \textbf{Hallar la transpuesta} del vector que tenemos $\Sigma_{yx}^{T}$

    $$
    \Sigma_{xy} = \Sigma_{yx}^{T} =
    \begin{pmatrix}
    3 & -1 \\
    3 & 2
    \end{pmatrix}
    $$

    \item \textbf{Calcular }$\Sigma_{yx} \Sigma_{xx}^{-1} \Sigma_{xy}$

    $$
    \Sigma_{yx} \Sigma_{xx}^{-1} \Sigma_{xy} =
    \begin{pmatrix}
    3 & 3 \\
    -1 & 2
    \end{pmatrix}
    \begin{pmatrix}
    \frac{7}{33} & \frac{3}{33} \\
    \frac{3}{33} & \frac{6}{33}
    \end{pmatrix}
    \begin{pmatrix}
    3 & -1 \\
    3 & 2
    \end{pmatrix}
    $$

    $$
    \Sigma_{yx} \Sigma_{xx}^{-1} \Sigma_{xy} =
    \begin{pmatrix}
    \frac{21}{33}+\frac{9}{33} & \frac{9}{33}+\frac{18}{33} \\
    -\frac{7}{33}+\frac{6}{33} & -\frac{3}{33}+\frac{12}{33}
    \end{pmatrix}
    \begin{pmatrix}
    3 & -1 \\
    3 & 2
    \end{pmatrix} =
    \begin{pmatrix}
    \frac{30}{33} & \frac{27}{33} \\
    -\frac{1}{33} & -\frac{9}{33}
    \end{pmatrix}
    \begin{pmatrix}
    3 & -1 \\
    3 & 2
    \end{pmatrix}
    $$

    $$
    \Sigma_{yx} \Sigma_{xx}^{-1} \Sigma_{xy} =
    \begin{pmatrix}
    \frac{90}{33}+\frac{81}{33} & -\frac{30}{33}+\frac{54}{33} \\
    -\frac{3}{33}+\frac{27}{33} & \frac{1}{33}+\frac{18}{33}
    \end{pmatrix} =
    \begin{pmatrix}
    \frac{171}{33} & \frac{24}{33} \\
    \frac{24}{33} & \frac{19}{33}
    \end{pmatrix}
    $$

    \item \textbf{Calcular: }$\text{cov}(\mathbf{y}|\mathbf{x}) =
\Sigma_{yy} - \Sigma_{yx} \Sigma_{xx}^{-1} \Sigma_{xy}$

    $$
    \text{cov}(\mathbf{y}|\mathbf{x}) =
    \begin{pmatrix}
    9 & 0 \\
    0 & 1
    \end{pmatrix}
    -
    \begin{pmatrix}
    \frac{171}{33} & \frac{24}{33} \\
    \frac{24}{33} & \frac{19}{33}
    \end{pmatrix}
    $$

    $$
    \text{cov}(\mathbf{y}|\mathbf{x}) =
    \begin{pmatrix}
    9-\frac{171}{33} & 0 - \frac{24}{33} \\
    0- \frac{24}{33} & 1 - \frac{19}{33} 
    \end{pmatrix} =
    \begin{pmatrix}
    \frac{126}{33} & -\frac{24}{33} \\
    -\frac{24}{33} & \frac{14}{33} 
    \end{pmatrix}
    
    $$

\end{enumerate}

\subsection{Correlación}

$\rho_{12}$

\subsubsection{Solución}

Tomando la fórmula de correlación se tiene 

$$
\rho_{12}=\frac{\sigma_{12}}{\sqrt{\sigma_{11}\sigma_{22}}}
$$

Teniendo la matriz de covarianzas

$$
\Sigma_{yy}=
\begin{pmatrix}
9 & 0 \\
0 & 1
\end{pmatrix} =
\begin{pmatrix}
\sigma_{11} & \sigma_{12} \\
\sigma_{21} & \sigma_{22}
\end{pmatrix}
$$

Sustituimos la relación que pide el problema por lo que se tiene

$$
\rho_{12}=\frac{0}{\sqrt{(9)(1)}}=0
$$

El coeficiente de correlación $\rho_{12}=0$ lo que indica que las variables $y_1$ y $y_2$ no están correlacionadas

\subsection{Correlación parcial}

$\rho_{12.34}$

\subsubsection{Solución}

Utilizando la fórmula de correlación parcial 

$$
\rho_{ij.rs \cdot \cdot \cdot q}=\frac{\sigma_{ij.rs \cdot \cdot \cdot q}}{\sqrt{\sigma_{ii.rs \cdot \cdot \cdot q}\sigma_{jj.rs \cdot \cdot \cdot q}}}
$$

Especificando para $q=4$ se tiene la siguiente forma

$$
\rho_{12.34}=\frac{\sigma_{12.34}}{\sqrt{\sigma_{11.34}\sigma_{22.34}}}
$$

Por lo cual se necesita calcular:\\

\begin{itemize}
    \item $\sigma_{11.34}=\sigma_{11}-\Sigma_{13}\Sigma_{33}^{-1} \Sigma_{31}$
    \item $\sigma_{22.34}=\sigma_{23}-\Sigma_{33}\Sigma_{33}^{-1} \Sigma_{32}$
    \item $\sigma_{11.34}=\sigma_{11}-\Sigma_{13}\Sigma_{33}^{-1} \Sigma_{32}$
\end{itemize}

Teniendo la matriz de covarianzas completa $\Sigma$

$$
\Sigma=
\begin{pmatrix}
9 & 0 & 3 & 3 \\
0 & 1 & -1 & 2 \\
3 & -1 & 6 & -3 \\
3 & 2 & -3 & 7 \\
\end{pmatrix}
$$

Se extraen las submatrices

$$
\Sigma_{33}=
\begin{pmatrix}
6 & -3 \\
-3 & 7
\end{pmatrix}
$$

$$
\Sigma_{13}=
\begin{pmatrix}
3 & 3 \\
0 & -1
\end{pmatrix}
$$

$$
\Sigma_{31}=\Sigma_{13}^T
\begin{pmatrix}
3 & 0 \\
3 & -1
\end{pmatrix}
$$

Ahora calculamos $\Sigma_{33}^-1$

$$
\Sigma_{33}^-1 = \frac{1}{det(\Sigma_{33})}
\begin{pmatrix}
7 & 3 \\
3 & 6
\end{pmatrix} = \frac{1}{33}
\begin{pmatrix}
7 & 3 \\
3 & 6
\end{pmatrix}
$$

Ahora calculamos los $\sigma_{11.34}$

$$
\sigma_{11.34} = \sigma_{11} - \Sigma_{13} \Sigma_{33}^{-1} \Sigma_{31}
$$

$$
\sigma_{11.34} = 9 - 
\begin{pmatrix}
3 & 3
\end{pmatrix} \frac{1}{33}
\begin{pmatrix}
7 & 3 \\
3 & 6
\end{pmatrix}
\begin{pmatrix}
3 \\
3
\end{pmatrix}
$$

$$
\sigma_{11.34} = \frac{126}{33}
$$

De la misma forma los otros valores

$$
\sigma_{22.34} = \frac{126}{33}
$$
$$
\sigma_{12.34} = -\frac{63}{33}
$$

Finalmente reemplazamos los valores en la ecuación y tenemos

$$
\rho_{12.34} = \frac{-\frac{63}{33}}{\sqrt{\frac{126}{33}\frac{126}{33}}} = -\frac{63}{126} = -0.5
$$















