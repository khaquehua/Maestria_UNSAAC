
\newpage

\section{Resumen de la investigación de interés}
El problema de investigación que estoy abordando es la detección del queratocono, diversos estudios indicaron que esta enfermedad esta asociada a diversos factores como por ejemplo: Antecedentes familiares, frotamiento de ojos, frecuencia de ojo seco, aumento y disparidad del defecto refractivo (astigmatismo) como otros. Frente a estas asociaciones se desarrollarón diferentes test para la detección temprana de esta enfermedad, como el señalado en la actividad (disparidad del astigmatismo corneal y refractivo) \cite{sinjab2021corneal}. Actualmente aún no se tiene una forma de detección segura de esta enfermedad, aunque se tienen equipos que ayuden a la detección más temprana de esta enfermedad como el MS-39 \cite{elkitkat2022accuracy}, la accesibilidad a estos equipos es más complicado en ciertas zonas. Adicional a esto varios estudios indican que la prevalencia de esta enfermedad es muy variado en diferentes partes del mundo, siendo más frecuente es zonas de mayor altura, por lo que es necesario hallar la prevalencia en la población de estudio.

\section{Metodología estadística}
La metodología estadística a usar para abordar el problema es desde el enfoque de la probabilidad, el análisis bayesiano y la regresión ya que se desea evaluar los diferentes métodos para la detección del queratocono, desde medir su sensibilidad y especificidad a través de la probabilidad condicional, evaluar el valor predictivo de cada test a través del teorema de bayes, asimismo usar el análisis bayesiano para combinar diferentes métodos o test utilizados para la detección de queratocono y actualizar la probabilidad de clasificación de la enfermedad. Por última parte se desea utilizar la regresión para evaluar la relación de las diferentes metodologías usadas, así como la agregación de variables de interés que puedan ayudar a la detección de esta enfermedad. \cite{peck2006statistics}

\section{Revisión de contribuciones de tendencia actuales}
Se encontraron los siguientes papers según los recursos y un resumen sobre el abordamiento del problema:
\begin{itemize}
    \item \textbf{Springer Journal Finder}
    \begin{itemize}
        \item Enhancing early detection of keratoconus suspects using interocular corneal tomography asymmetry \cite{li2025enhancing}: Mediante el índice de asimetría interocular clasifica los ojos con sospecha de queratocono, queratonoco y normales a través de los índices de la tomografía corneal y regresión logística realizando el análisis de la sensibilidad y especificadad con la curva ROC. En resultados se aplicó test paramétricos y no paramétricos para ver si existen diferencias estadísticamente significativas en los parámetros tomográficos, viendo que la asimetría corneal ayuda en la sospecha de queratocono inicial.

        \item Evaluation of parameters for early detection of pediatric keratoconus \cite{zhao2024evaluation}: Mostraron que la progresión del queratocono es más frecuente en población infantil y joven, por lo que es necesario detectarlo en ese lapso de tiempo. Tomando los índices de las imágenes de Scheimpflug dando las clasificaciones de córneas normales y córneas con queratocono. Se mostró que los índices de PENTACAM fueron anormales en la detección y exclusión de pacientes con queratocono. La metodología fue la comparación de grupos y ver si hay diferencias significativas entre los grupos de estudio.
    \end{itemize}
    \item \textbf{Elsevier Journal Finder}
    \begin{itemize}
        \item Thickness Speed Progression Index: Machine Learning Approach for Keratoconus Detection \cite{awwad2025thickness}:  Tuvo como objetivo desarrollar un modelo de machine learning basado en la paquimetría para la diferenciación de queratocono, sospecha de queratocono y corneas normales. Teniendo a técnicas como el Random Forest con una precisión del $100\%$ para la detección entre ojos con queratocono y normales, una precisión del $91\%$ para la detección entre ojos con sospecha, queratocono y normales y una precisión del $87\%$ para ojos normales, queratocono, tomograficamente normal y ojos con compañeros limitrofes. 
    \end{itemize}
    \item \textbf{Otras fuentes}
    \begin{itemize}
        \item Algo interesante que se encontró es el siguiente enlace con un pre-trabajo que se encuentra en revisión \href{https://arxiv.org/html/2501.12531v1}{The BAD Paradox: A Critical Assessment of the Belin/Ambrósio Deviation Model}. En el cual se explica que la forma de clasificación del BAD-D PENTACAM utiliza una regresión lineal, asimismo cuando se aplicó el estudio en su población se mostró que no se cumplen la mayor parte de supuestos que debe cumplir una regresión lineal.
    \end{itemize}
\end{itemize}

\section{Identificación de la investigación}

La investigación es de tipo aplicada ya que se evalúa los resultados de los diferentes test y parámetros y ver la influencia que puedan tener con un enfoque cuantitativo.

El alcance es descriptivo y correlacional teniendo como variable dependiente el caso de un ojo si es con queratocono o cornea normal y variables independientes los test y parámetros.

Se tiene un diseño no experimental ya que solo se observaron y analizaron eventos pasados anteriormente (transversal).

\section{Identificación del problema de investigación}

El problema de investigación planteada sería ¿Cómo mejorar la detección de queratocono teniendo en cuenta diferentes test, estudios, parámetros? que se desea resolver a través de la probabilidad, análisis bayesiano y regresión.

La justificación es por la aplicación de métodos estadísticos asimismo como ir más alla de la significancia de asociación y relación. También que se desea que la población de estudio sea la sureña del Perú (Cusco) y evaluar su prevalencia, asimismo algo importante a resaltar es que se debe y mostrar in sito la metodología correcta y uso de los diferentes métodos estadísticos para tener los resultados más confiables.

El estudio aportará tanto en la parte de la estadística aplicada (al mostrar la base sólida con la técnica a utilizar como el cumplimiento de supuestos en la regresión, asimismo en el modelamiento matemático de la enfermedad del queratocono a través de las probabilidades), impacto social (concientización desde la prevalencia del queratocono a los entes encargados de salud tomando en cuenta que la población de estudio se expone a una gran altitud, realizar un test más accesible a insumos disponibles que tiene la mayor parte de centros de salud).

\section{Socialización con el asesor}
Proceso

