
\newpage

\section{Planteamiento del problema}
El queratocono es una enfermedad que afecta a la córnea causando mala visión de forma progresiva y evolutia en pacientes jóvenes. Según la Organización Mundial de la Salud (OMS) esta enfermedad puede indicar cifras epidémicas. A pesar de que se considere una afección de impacto moderado y no tan grave que afecta a la función visual y calidad de vida, esta enfermedad puede resultar más perjudicial ya que puede afectar a individuos totalmente activos \cite{thalasselis1988keratoconus}. El queratocono generalmente inicia en la pubertad hasta la tercera o casos de cuarta década de vida \cite{althomali2018prevalence}, sus síntomas son variables y dependen de la etapa de progresión que se encuentre, en etapas iniciales no pueden presentarse síntomas hasta etapas más avanzadas en donde el paciente experimenta una distorsión notable en la visión acompañada de una pérdida visual severa. A pesar de eso no llegan a producir ceguera completa debido a esta enfermedad \cite{rabinowitz1998keratoconus}. 

Los primeros estudios de prevalencia se llevó en la población de Indianapolis, USA encontró 600 casos por 100,000 (0.6$\%$). Un estudio poblacional más reciente informó una prevalencia de 375 por 100,000 (0.375$\%$) en Países Bajos \cite{godefrooij2017age}. Mientras que en Latino América  los registros de prevalencia son muy pocos, teniendo en Paraná entre Ríos-Argentina una prevalencia aproximada de 229 por cada 100,000 habitantes con edad promedio de diagnóstico a los 24.5 años \cite{pussetto2011alta}, en la ciudad de Quito en Ecuador se tuvo una prevalencia de (0.75$\%$) \cite{mansfield2017queratocono}, por otra parte en el Hospital Regional Honorio Delgado, durante el periodo 2014 - 2017 en Arequipa-Perú se evidenció que la prevalencia de queratocono fue de 1.37 por cada mil pacientes atendidos \cite{ramos2018prevalencia}. Por último con respecto a la ciudad del Cusco en el trabajo de investigación de \cite{samaniego2023prevalencia} reportó en el Centro de Salud Integral La Fuente una prevalencia de 3.32$\%$ siendo esta mayor a las otras zonas estudiadas en el cual aún no se le esta dando la importancia por parte del estado peruano.

Asimismo en la clínica Oftalmo Salud del Perú indica que no se sabe con certeza porque algunas personas desarrollan Queratocono, aunque hay estudios con asociaciones a la genética o dicho hereditaria, asimismo con alergías y lesiones oculares provocado por la frotación excesiva de los ojos, el uso inapropiado de lentes de contacto y la presencia de algunas patologías como la queratoconjuntivis vernal, la retinitis pigmentaria o la retinopatía del prematura. Múltiples estudios han desarrollado técnicas para poder realizar la detección temprana de esta enfermedad ya sea por modelos de Machine Learning que indicaron que índices de PENTACAM fueron anormales para detección y exclusión de pacientes con queratocono \cite{zhao2024evaluation}, usando la regresión logística tomando los índices de asimetría \cite{li2025enhancing}.

En este caso va la importancia del correcto de modelos estadísticos

